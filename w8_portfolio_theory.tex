% !TeX spellcheck = en_US
\documentclass[11pt]{article}
\usepackage{exerciseHead}

\input{Macros.tex} 

% Meta Information %%%%%%%%%%%%%%%%%%%%%%%%%%%%%%%%%%%

% included in all sheets (-> only one update per semester)
\course{Stochastik 1}
\tutorial{Gruppenarbeitstutorium}
\semester{HWS 2022}
\author{Felix Benning, Svenja Kaiser}
 % (file) for one update per semester

\title{Portfolio Theory}

%%%% Document %%%%%%%%%%%%%%%%%%%%%%%%%%%%%%%%%%%%%%%%
\begin{document}

\maketitle

\section{Modern Portfolio Theory}

\begin{exercise}[Diversification]
	Let \(R^{(1)},R^{(2)},\dots\) be random variables representing the stock
	returns of different companies. Assume that the variance of the stock returns
	are all equal \(\Var(R^{(i)}) = 1\), and that the stock returns are correlated
	positively \(\Cov(R^{(i)},R^{(j)}) = \sigma^2 < 1\), which we call market volatility.

	Assume \(\sigma^2=0.5\). How many stocks need to be in an equal weight
	portfolio, until the variance of the portfolio is smaller than \(0.6\)?
	Can the variance of the portfolio become arbitrarily small?
\end{exercise}
\begin{hint}
	Bienaymé	
\end{hint}

\section{Black-Scholes}

Instead of modelling multiple companies in one time step, we now want to model
a single stock over time. The most famous model for that, is the Black-Scholes
model.  It is based on the Geometric Brownian Motion
\[
	S_t = S_0\exp( (\mu - \tfrac{\sigma^2}{2}) t + \sigma W_t),
\]
where \(W_t\) is a Brownian Motion over time \(t\ge 0\), with \(\mu\ge0\) and
\(\sigma > 0\). Since stochastic processes (including the Brownian Motion) are
too advanced for now, we are going to discretize the Brownian motion to a random
walk
\[
	W_t = \sum_{k=0}^{t-1} \Delta W_k
\]
where \(\Delta W_k\overset{iid}\sim\Normal(0,1)\) and restrict ourselves to \(t\in\nat\).

\begin{exercise}[Expected Return]
	Calculate the expected return \(\E[R_t]\) for time \(t\), where \(R_t=S_t/S_0\).
\end{exercise}

\begin{exercise}[Geometric Average]
	Find the (random) interest rate \(\bar{R}_t\), such that investing \(S_0\)
	at this interest rate for \(t\) years is equal to \(S_t\), i.e.
	\[
		(\bar{R}_t)^t S_0 = S_t.
	\]
	Argue, why this random geometric average should stabilize for \(t\to\infty\).
\end{exercise}
\begin{hint} The \(\log\) function.
\end{hint}

\begin{exercise}[Concentration Inequalities]
	Let \(\sigma=1\), \(\mu=0.1\). For how long do you have to be exposed to the
	stock for the risk free alternative interest \(r=0.03\) to be worse than the
	investment into the stock with at least \(90\%\) probability? I.e. find 
	the smallest \(t\), such that
	\[
		\Pr(\bar{R}_t > r) \ge 0.9.
	\]
\end{exercise}

\end{document}
