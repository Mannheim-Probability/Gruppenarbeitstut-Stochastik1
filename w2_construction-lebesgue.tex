% !TeX spellcheck = en_US
\documentclass[11pt]{article}
\usepackage{exerciseHead}

\input{Macros.tex} 

% Meta Information %%%%%%%%%%%%%%%%%%%%%%%%%%%%%%%%%%%

% included in all sheets (-> only one update per semester)
\course{Stochastik 1}
\tutorial{Gruppenarbeitstutorium}
\semester{HWS 2022}
\author{Felix Benning, Svenja Kaiser}
 % (file) for one update per semester

\title{Lebesgue-Maß}

%%%% Document %%%%%%%%%%%%%%%%%%%%%%%%%%%%%%%%%%%%%%%%
\begin{document}

\maketitle

\begin{exercise}[Konstruktion]
	Auf dem Blatt zur ``Intuition von Maßräumen'' haben wir schon skizziert,
	warum die Eigenschaften
	\begin{itemize}
		\item \(\lambda(\emptyset) = 0, \lambda(A)\ge 0\) für alle \(A\in\cB(\real)\)
		\item \(\lambda(\biguplus_{k=1}^\infty A_k) = \sum_{k=1}^\infty \lambda(A_k)\)
		``Approximierbarkeit''
		\item \(\lambda(A+x) = \lambda(A)\) für alle \(x\in\real^n\)
		``Verschiebungsinvarianz''
		\item \(\lambda([0,1]^n) = 1\) ``Normierung''
	\end{itemize}
	das Lebesgue Maß eindeutig definieren. Allerdings haben wir bei den Sigma
	Algebren ein bisschen geschludert weil wir die Intuition betont haben
	und auch nicht die Existenz gezeigt, sondern nur die Eindeutigkeit skizziert.
	
	In dieser Aufgabe geht es darum diese Lücken zu füllen. Dazu wollen wir
	auch den Fortsetzungssatz von Carathéodory (Satz 1.3.7) anwenden. Der Satz wird
	zwar erst in am Anfang der Vorlesung 5 eingeführt, aber um ihn zu verwenden
	braucht ihr nur die Objekte in den Voraussetzungen des Satzes zu konstruieren,
	die alle schon vorher eingeführt wurden.

	Daher ist die Aufgabe: Beweise dass es genau ein Maß gibt (Existenz und
	Eindeutigkeit) welches die oben genannten Eigenschaften erfüllt.
\end{exercise}

\begin{exercise}[Verzerrung]
	Sei \(f:\real^n\to\real^n\) eine lineare Abbildung. Dann definieren wir
	\[
		\lambda_f(A) := \lambda(f^{-1}(A))
	\]
	als das ``Push-Forward'' Maß oder ``Bildmaß'' von \(\lambda\) mit \(f\).
	(Das geht auch allgemeiner für ``messbare'' Funktionen, die in Kapitel 2
	der Vorlesung in Satz 2.2 eingeführ werden.)

	Zeigt
	\[
		\lambda_f(A) = \frac1{|\det(f)|}\lambda(A).
	\]
\end{exercise}
\begin{remark}
	Warum ist das interessant? Zum einen sieht man hier, warum man die
	Determinante als ein Maß für die Volumenverzerrung sehen kann, die eine
	lineare Funktion verursacht. Hat die lineare Abbildung nicht vollen Rang,
	bildet sie einen \(n\)-Dimensionalen raum in einen Unterraum ab. Also
	verschwindet das Volumen, da eine (Hyper-)ebene kein Volumen hat. Eine
	nicht-invertierbare Matrix hat also Determinante Null.

	Da differenzierbare Funktionen lokal wie eine lineare Funktionen aussehen
	(die Ableitung), kann man mit dieser Aussage außerdem eine konkrete Version
	des abstrakten Transformationssatzes (Satz 3.1.16) basteln. Diese konkrete
	Version ist bekannter als ``Substitutionsregel''.
\end{remark}

\begin{hint}
	Zeigt \(\lambda_f\) ist Verschiebungsinvariant. Zeigt es für \(f\), die
	isomorph zu einer Diagonalmatrix sind, und verallgemeinert dann.
\end{hint}
\begin{hint}
	Zeigt oder erinnert euch, dass \(\det(AB)=\det(A)\det(B)\). Folgt daraus,
	dass für orthogonale Matrizen \(U\) gilt \(\det(U)=\pm 1\). Zuletzt, wendet
	die Singulärwertzerlegung einer Matrix an.
\end{hint}


\end{document} 
