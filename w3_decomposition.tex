% !TeX spellcheck = en_US
\documentclass[11pt]{article}
\usepackage{exerciseHead}

\input{Macros.tex} 

% Meta Information %%%%%%%%%%%%%%%%%%%%%%%%%%%%%%%%%%%

% included in all sheets (-> only one update per semester)
\course{Stochastik 1}
\tutorial{Gruppenarbeitstutorium}
\semester{HWS 2022}
\author{Felix Benning, Svenja Kaiser}
 % (file) for one update per semester

\title{Zerlegung}

%%%% Document %%%%%%%%%%%%%%%%%%%%%%%%%%%%%%%%%%%%%%%%
\begin{document}

\maketitle

\begin{exercise}[Maße]
	Sei \((\Omega,\cA,\mu)\) ein Maßraum.
	\begin{enumerate}
		\item 
		Zeige, dass wenn \(\mu\) ein eindliches Maß ist, für alle \(\epsilon>0\)
		höchstens endlich viele Punkte \(x\) existieren mit \(\mu(\{x\})>\epsilon\).

		\item Sei \(E\) eine messbare Menge mit \(\mu(E)<\infty\). Zeige, dass dann
		\(\eta(A):=\mu(A\cap E)\) für \(A\in\cA\) ein endliches Maß ist.

		\item Folgere, dass ein \(\sigma\)-endliches Maß höchstens abzählbar viele
		Punkte \(x\) mit positiver Masse \(\mu(\{x\})>0\) besitzt.
	\end{enumerate}
\end{exercise}

\begin{exercise}[Funktionen]
	\leavevmode
	\begin{enumerate}
		\item Zeige, dass eine Verteilungsfunktion \(F\) höchstens abzählbar viele
		``Sprungstellen'' (bzw. ``Unstetigkeitsstellen'') besitzt. Wobei eine
		Sprungstelle \(x\) über die folgende Eigenschaft definiert ist:
		\[
			\lim_{y\uparrow x}F(x)\neq \lim_{y\downarrow x} F(y).
		\]


		\item Sei \(f\) eine beliebige monotone Funktion. Zeige, dass es für jedes
		Intervall \([a,b]\) Konstanten \(c,d\in\real\) gibt, sodass
		\[
			F(x):= \begin{cases}
				0 & x\le a\\
				\lim_{y\downarrow x} cf(y) + d & x\in [a,b]\\
				1 & x\ge b
			\end{cases}
		\]
		eine Verteilungsfunktion ist.

		\item Folgere, dass jede monotone Funktion höchstens abzählbar viele
		Unstetigkeitsstellen besitzt.
	\end{enumerate}	
\end{exercise}

\begin{exercise}[Zerlegung]
	Sei \(\mu\) ein Wahrscheinlichkeitsmaß auf \(\real\) mit Verteilungsfunktion
	\(F\). Zeige dass es zwei Wahrscheinlichkeitsmaße \(\mu_d,\mu_s\) mit
	Verteilungsfunktionen \(F_d, F_s\) gibt, sodass 
	\begin{enumerate}
		\item \(F_s\) eine stetige Funktion ist,

		\item \(\mu_d\) ein diskretes Maß ist, i.e.
		\[
			\mu_d = \sum_{k=1}^\infty a_k \delta_{x_k},
			\qquad (x_i)_{i\in\nat}, (a_i)_{i\in\nat} \subseteq \real,
		\]

		\item und zuletzt gilt, dass
		\[
			\mu = c_d \mu_d + c_s \mu_s \quad\text{und}\quad F = c_d F_d + c_s F_s
		\]
		für \(c_d,c_s\in[0,1]\).
	\end{enumerate}
\end{exercise}

\end{document} 
