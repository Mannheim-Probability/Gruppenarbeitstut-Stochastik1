% !TeX spellcheck = en_US
\documentclass[11pt]{article}
\usepackage{exerciseHead}

\input{Macros.tex} 
\titleformat*{\section}{\Large\bfseries}

% Meta Information %%%%%%%%%%%%%%%%%%%%%%%%%%%%%%%%%%%

% included in all sheets (-> only one update per semester)
\course{Stochastik 1}
\tutorial{Gruppenarbeitstutorium}
\semester{HWS 2022}
\author{Felix Benning, Svenja Kaiser}
 % (file) for one update per semester

\title{Continuous Mapping}

%%%% Document %%%%%%%%%%%%%%%%%%%%%%%%%%%%%%%%%%%%%%%%
\begin{document}

\maketitle

\section{Almost Sure Convergence}


\begin{exercise}
	Let \(f\) be continuous up to a \(\Pr_X\) zero set. Prove
	\[
		X_n\overset{a.s.}\to X \implies f(X_n)\overset{a.s}\to f(X).
	\]
\end{exercise}

\begin{exercise}
	Let \(X_n^{(i)}\overset{a.s.}\to X^{(i)}\) for \(i\in\{1,\dots, m\}\).
	Prove that we have
	\[
		(X_n^{(1)},\dots, X_n^{(m)}) \overset{a.s.}{\to} (X^{(1)},\dots, X^{(m)}).
	\]
\end{exercise}
\begin{hint}
	In finite dimesion all norms are equivalent.
\end{hint}

\begin{exercise}[Application]
	Let \(X_n\overset{iid}\sim\text{Geo}(p)\) be geometrically distributed for
	\(p>0\). We know by the strong law of lare numbers, that
	\[
		\bar{X}_n := \frac1n\sum_{k=1}^n X_k \overset{a.s.}\to \E[X_1] = \frac1p.
	\]
	Construct an ``a.s. consistent estimator'' \(\hat{P}_n\) as a function of
	\(X_1\dots, X_n\), where consistence implies 
	\[
		\hat{P}_n \overset{a.s.}\to p.
	\]
\end{exercise}


\section{\texorpdfstring{\(L^p\)}{Lp} Convergence}

\begin{exercise}\label{ex: Lp cmt}
	Prove that if either of these statements hold
	\begin{enumerate}
		\item \(f\) is Lipschitz continuous
		\item \(f\) is continuous and bounded,
	\end{enumerate}
	then \(X_n \overset{L^p}\to X\) implies \(f(X_n)\overset{L^p}\to f(X)\).
	You also need to show that \(f(X_n),f(X)\in L^p\).
\end{exercise}

\begin{exercise}
	Let \(\Omega=[0,1]\) with \(\Pr=\restr{\lambda}{[0,1]}\) and define the random variables
	\[
		X_n :=\sqrt{n}\mathbbm{1}_{[0,1/n]}.
	\]
	Prove that \(X_n\overset{L^1}\to 0\), but for \(g(x):= x^2\) we have
	\(g(X_n) \overset{L^1}{\not\to} g(0)=0\). Why is this not a contradiction to
	Exercise~\ref{ex: Lp cmt}?
\end{exercise}

\begin{exercise}
	Prove for iid \(X_i\)
	\[
		\frac1{n-1}\sum_{i=1}^n (X_i - \bar{X}_n) \overset{L^2}\to \Var(X_1) \quad n\to\infty.
	\]
\end{exercise}
\begin{hint}
	Calculate expectation and variance (no CM necessary).
\end{hint}

\section{Convergence in Probability}

\begin{exercise}[Difficult]
	Let \(f\) be continuous up to a \(\Pr_X\) zero set. Prove
	\[
		X_n\overset{p}\to X \implies f(X_n)\overset{p}\to f(X).
	\]
\end{exercise}
\begin{hint}
	Prove that \(B_{m+1}^\complement \subseteq B_m^\complement\) for
	\[
		B_m := \{x\in\real^d : \forall y\in B_{\frac1m(x)} : \| f(x)- f(y)\| < \epsilon\},
	\]
	and that \(\bigcap_{m\in\nat} B_m^\complement\) is a \(\Pr_X\) zero set. What
	does this imply for \(\lim_{m\to\infty}\Pr(X\not\in B_m)\)? Use the ``good
	set principle'' to finish the proof.
\end{hint}


\begin{exercise}
	Let \(X_n^{(i)}\overset{p}\to X^{(i)}\) for \(i\in\{1,\dots, m\}\).
	Prove that 
	\[
		(X_n^{(1)},\dots, X_n^{(m)}) \overset{p}{\to} (X^{(1)},\dots, X^{(m)}).
	\]
\end{exercise}
\begin{hint}
	Sup-norm.
\end{hint}

\begin{exercise}
	Prove for iid \(X_i\) with \(\Var(X_1) = \sigma^2\)
	\[
		\sqrt{\frac1{n}\sum_{i=1}^n (X_i - \bar{X}_n)} \overset{p}\to \sigma \quad n\to\infty.
	\]
\end{exercise}
\section{Convergence in Distribution}

\begin{exercise}\label{ex: conv in dist cmt}
	For a continuous function \(f\), prove that \(X_n\overset{d}\to X\) implies
	\(f(X_n)\overset{d}\to f(X)\).
\end{exercise}

\begin{exercise}
	Explain why it is not true in general, that
	\[
		X_n\overset{d}\to X \text{ and } Y_n\overset{d}\to Y
		\implies X_n + Y_n \overset{d}\to X+Y.
	\]
	Why is this not a contradiction to Exercise~\ref{ex: conv in dist cmt}?
\end{exercise}
\begin{hint}
	\(X_n := Y := X \sim \Normal(0,1)\) and \(Y_n:=-X_n\).
\end{hint}

\begin{exercise}
	Prove that \(X_n\overset{d}\to X\) and \(Y_n\overset{p}\to c\in\real\)
	implies \((X_n,Y_n)\overset{d}\to (X,c)\).
\end{exercise}

\begin{exercise}
	Let \(X_i\) be iid with \(\E[X_1]=0\) and \(\Var(X_1)=\sigma^2\). Then by the
	central limit theorem
	\[
		\frac{1}{\sqrt{n}\sigma} \sum_{k=1}^n X_i \overset{d}\to \Normal(0,1).
	\]
	Deduce
	\[
		\frac{\sum_{k=1}^n X_i}{\sqrt{\sum_{i=1}^n (X_i - \bar{X}_n)^2}}
		\overset{d}\to \Normal(0,1).
	\]
\end{exercise}


\end{document}