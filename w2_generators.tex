% !TeX spellcheck = en_US
\documentclass[11pt]{article}
\usepackage{exerciseHead}

\input{Macros.tex} 

% Meta Information %%%%%%%%%%%%%%%%%%%%%%%%%%%%%%%%%%%

% included in all sheets (-> only one update per semester)
\course{Stochastik 1}
\tutorial{Gruppenarbeitstutorium}
\semester{HWS 2022}
\author{Felix Benning, Svenja Kaiser}
 % (file) for one update per semester

\title{Erzeuger}

%%%% Document %%%%%%%%%%%%%%%%%%%%%%%%%%%%%%%%%%%%%%%%
\begin{document}

\maketitle

\begin{exercise}[Metrischer Raum]\label{ex: metrischer Raum}
	Sei \((X,d)\) ein metrischer Raum. Ein Ball ist definiert als
	\[
		B_\epsilon(x) := \{y\in X: d(x,y) < \epsilon\}.
	\]
	Eine offene Menge \(O\) besitzt die definierende Eigenschaft
	\[
		\forall x\in O, \exists \epsilon >0 : B_\epsilon(x) \subseteq O.
	\]
	Sei \(\tau\) die Topologie (Menge der offenen Mengen) von \(X\) induziert
	durch \(d\). Dann ist die Borel \(\sigma\)-Algebra definiert als
	\[
		\cB(X):=\sigma(\tau).
	\]
	Nehme nun an, dass der metrische Raum \(X\) separabel ist, mit abzählbarer
	dichter Teilmenge \(Q\). Zeige, dass dann
	\[
		\cE:= \{ B_{\frac1n}(x): x\in Q, n\in\nat \}
	\]
	ein Erzeuger von \(\cB(X)\) ist. Folgere, dass die Menge der Würfel
	\(\cB(\real^n)\) erzeugen.
\end{exercise}

\begin{exercise}[Urbilder]
	Für eine beliebige Funktion \(f\) und Mengensystem \(\cE\), zeige
	\[
		\sigma(f^{-1}(\cE)) = f^{-1}(\sigma(\cE)).
	\]	
	Folgere, dass stetig invertierbare Funktionen Borelmengen auf Borelmengen
	abbilden.
\end{exercise}

\begin{exercise}[Spur-\(\sigma\)-Algebra]
	Für eine beliebige Menge \(A\) und Mengensystem \(\cE\) ist zu zeigen
	\[
		\sigma(\cE \cap A) = \sigma(\cE) \cap A.
	\]
	wobei wir eine Menge mit einem Mengensystem schneiden, indem wir alle
	Elemente schneiden
	\[
		\cE \cap A := \{ E\cap A : E\in \cE\}.
	\]
	Folgere, dass der Schnitt einer \(\sigma\)-Algebra mit einer Menge wieder
	eine Sigma Algebra ist.
\end{exercise}

\begin{exercise}[Kartesisches Produkt]
	Seien \(\cE_1,\cE_2\) Mengensysteme mit \(\cA_i:=\sigma(\cE_i)\) und der
	Eigenschaft, dass jeweils Folgen \((E_i^{(k)})_{k\ge 1}\subseteq \cE_i\)
	existieren mit \(\bigcup_{k=1}^\infty E_i^{(k)} = \Omega_i\). Zeige, dass
	dann gilt
	\[
		\sigma(\cE_1 \times \cE_2) = \sigma(\cA_1\times \cA_2).
	\]
	Erkläre, wie man diese Aussage zusammen mit Aufgabe~\ref{ex: metrischer Raum}
	verwenden kann, um viele der Erzeugendensysteme von \(\cB(\real^n)\) aus den
	Erzeugendensystemen von \(\cB(\real)\) herzuleiten.
\end{exercise}
\begin{hint}
	Zeige für die Projektionsabbildungen \(\pi_i: \Omega_1\times\Omega_2 \to
	\Omega_i, x\mapsto x_i\), dass gilt
	\[
		\begin{aligned}
			\sigma(\cE_1 \times \cE_2)
			&\subseteq \sigma(\cA_1\times \cA_2)\\
			&\subseteq \sigma(\pi_1^{-1}(\cA_1)\cap \pi_2^{-1}(\cA_2))\\
			&\subseteq \sigma(\pi_1^{-1}(\cA_1)\cup \pi_2^{-1}(\cA_2))\\
			&\subseteq \sigma(\pi_1^{-1}(\cE_1)\cup \pi_2^{-1}(\cE_2))\\
			&\subseteq \sigma(\cE_1 \times \cE_2)
		\end{aligned}
	\]
\end{hint}

\end{document} 
