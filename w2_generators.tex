% !TeX spellcheck = en_US
\documentclass[11pt]{article}
\usepackage{exerciseHead}

\input{Macros.tex} 

% Meta Information %%%%%%%%%%%%%%%%%%%%%%%%%%%%%%%%%%%

% included in all sheets (-> only one update per semester)
\course{Stochastik 1}
\tutorial{Gruppenarbeitstutorium}
\semester{HWS 2022}
\author{Felix Benning, Svenja Kaiser}
 % (file) for one update per semester

\title{Erzeuger}

%%%% Document %%%%%%%%%%%%%%%%%%%%%%%%%%%%%%%%%%%%%%%%
\begin{document}

\maketitle

\begin{exercise}[\(\cB(\real)\) Erzeuger]
	Bei Definition sind wird die Borel Sigma Algebra von den offenen Mengen
	erzeugt \(\cB(\real)=\sigma(\cE_0)\). Zeige, dass all diese Mengensysteme
	die Borel Sigma Algebra erzeugen
	\[
		\begin{aligned}
		\cE_0 &= \{O\subseteq \real : O \text{ offen}\}\\
		\cE_1 &= \{A\subseteq \real : A \text{ abgeschlossen}\}\\
		\cE_2 &= \{K\subseteq \real : K \text{ kompakt}\}\\
		\cE_3 &= \{ (a, b) : a,b\in \real \}\\
		\end{aligned}
		\qquad
		\begin{aligned}
		\cE_4 &= \{ (a, b) : a,b\in\rational\}\\
		\cE_5 &= \{ [a, b) : a,b\in\rational\}\\
		\cE_6 &= \{ (-\infty, b] : a,b\in\rational\}\\
		\cE_7 &= \{ (-\infty, b) : a,b\in\rational\}\\
		\end{aligned}
	\]
\end{exercise}

\begin{exercise}
	Für eine beliebige Funktion \(f\) und Mengensystem \(\cE\), zeige
	\[
		\sigma(f^{-1}(\cE)) = f^{-1}(\sigma(\cE))
	\]	
\end{exercise}

\begin{exercise}[Spur-\(\sigma\)-Algebra]
	Für eine beliebige Menge \(A\) und Mengensystem \(\cE\) ist zu zeigen
	\[
		\sigma(\cE \cap A) = \sigma(\cE) \cap A.
	\]
	wobei wir eine Menge mit einem Mengensystem schneiden, indem wir alle
	Elemente schneiden
	\[
		\cE \cap A := \{ E\cap A : E\in \cE\}.
	\]
	Folgere, dass der Schnitt einer \(\sigma\)-Algebra mit einer Menge wieder
	eine Sigma Algebra ist.
\end{exercise}

\end{document} 
