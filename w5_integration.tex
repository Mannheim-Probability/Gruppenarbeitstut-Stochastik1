% !TeX spellcheck = en_US
\documentclass[11pt]{article}
\usepackage{exerciseHead}

\input{Macros.tex} 
\pagenumbering{arabic}

% Meta Information %%%%%%%%%%%%%%%%%%%%%%%%%%%%%%%%%%%

% included in all sheets (-> only one update per semester)
\course{Stochastik 1}
\tutorial{Gruppenarbeitstutorium}
\semester{HWS 2022}
\author{Felix Benning, Svenja Kaiser}
 % (file) for one update per semester
\tutorial{Merkblatt}
\author{Felix Benning}

\title{Integration}

%%%% Document %%%%%%%%%%%%%%%%%%%%%%%%%%%%%%%%%%%%%%%%
\begin{document}

\maketitle

\section{Riemann Integral}

\subsection{Hauptsatz der Analysis}

\begin{theorem}[Hauptsatz I]
	Sei
	\[
		F(x) = \int_a^x f(t) dt,
	\]	
	für ein stetiges \(f\). Dann ist \(F\) stetig differenzierbar
	mit \(F'=f\).
\end{theorem}
\begin{proof}
	\[
		F^{\delta}(x):=\frac{F(x+\delta) - F(x)}{\delta}
		= \frac1\delta \int_x^{x+\delta} f(t)dt
		\le \frac1\delta
		\int_x^{x+\delta}\sup_{s\in[x,x+\delta]} f(s)dt
		= \sup_{s\in[x,x+\delta]} f(s)
	\]
	Analog bekommt man die untere Schranke, mit der man mithilfe des
	Sandwichtheorems den Beweis beenden kann:
	\begin{align*}
		f(x)
		\overset{\text{stetig}}&= \liminf_{\delta\to 0}
		\inf_{s\in[x,x+\delta]} f(x)
		\le \liminf_{\delta\to 0} F^\delta(x)\\
		&\le F'(x)\\
		&\le \limsup_{\delta\to 0} F^\delta(x)
		\le \limsup_{\delta\to 0} \sup_{s\in[x,x+\delta]} f(x)
		\overset{\text{stetig}}= f(x).
		\qedhere
	\end{align*}
\end{proof}

\begin{theorem}[Hauptsatz II]
	Sei \(F\) differenzierbar mit \(F'=f\) für ein setiges \(f\). Dann gilt
	\[
		\int_a^b f(t)dt = F(x)\Bigl|_a^b := F(b) - F(a) 
	\]	
\end{theorem}
\begin{proof}
	Wir haben \(F'=f\), definiere \(G\) wie \(F\) im Hauptsatz I. Dann gilt
	nach dem Hauptsatz I, dass \(G'=f=F'\). Also gilt \((F-G)'=0\). Daraus
	folgt
	\begin{align*}
		F(b) - F(a)
		&= (F-G)(b) - (F-G)(a) + (G(b)-G(a))\\
		\overset{\text{\color{red} MWS}}&=
		\underbrace{(F-G)'(\xi)}_{=0} + G(b)-G(a)\\
		\overset{\text{def}}&= \int_a^b f(x) dx - \underbrace{\int_a^a f(x)dx}_{=0}.
		\qedhere
	\end{align*}
\end{proof}
\begin{remark}[Stückweise Setig/Differenzierbar]
	Mit der Linearität des Integrals, können diese Aussagen auch auf stückweise
	stetige \(f\), bzw. stückweise differenzierbare \(F\) ausgeweitet werden.
\end{remark}

\begin{example}[\(\sin\) Integration]\label{ex: sin integration}
	Zeige
	\[
		\int_0^x\sin(t)dt = 1-\cos(x)
	\]
\end{example}
\begin{proof}
	Es gilt \((-\cos)'(x) = \sin(x)\). Also gilt nach dem zweiten Hauptsatz
	\[ 
		\int_0^x\sin(t)dt
		= (-\cos)(x) - (-\cos(0)) = 1-\cos(x).
		\qedhere
	\]
\end{proof}

\subsection{Uneigentliche Integrale}

Es gibt immer wieder Fälle, in denen der Hauptsatz II nicht anwendbar ist, weil
\(a\) und/oder \(b\) nicht in \(F\) einsetzbar ist, z.B. weil \(a=-\infty\)
oder \(b=-\infty\), oder weil \(F\) dort eine
Polstelle hat. In diesen Fällen, greifen wir zu uneigentlichen Integralen.
Ein uneigentliches Integral für \(a\in\real\), \(a\le b \le \infty\) ist
definiert als
 \[
	\int_a^b f(x)dx
	:= \lim_{\beta\uparrow b}\int_a^\beta f(x)dx
	= \lim_{\beta\uparrow b} F(x)\Bigl|_a^\beta
	=: F(x)\Bigl|_a^b
\]
Und analog für die untere Grenze. Falls beide Grenzen problematisch sind, teilt
man das Integral in zwei Teile. Konvergieren beide Seiten, kann man das
Integral wieder zusammensetzen:
 \[
	\int_a^b f(x)dx
	:= \lim_{\beta\uparrow b}\int_0^\beta f(x)dx
	+ \lim_{\alpha\downarrow a}\int_\alpha^0 f(x)dx
	= \underbrace{
		\lim_{\beta\uparrow b} F(x)\Bigl|_0^\beta
		+ \lim_{\alpha\downarrow a} F(x)\Bigl|_\alpha^0
	}_{
		=\lim_{\beta\uparrow b} F(\beta)
		- \lim_{\alpha\downarrow a} F(\alpha)
	}
	=: F(x)\Bigl|_a^b
\]
\begin{example}[\(\exp\) Integration]\label{ex: exp integration}
	Sei \(c\ge 0\), dann gilt
	\[
		\int_0^\infty \exp(-ct)dt = \lim_{x\to\infty}-\frac1c\exp(-ct)\Bigl|_0^x
		= -\left(0-\frac1c\right)
		=\frac1c.
	\]	
\end{example}

\clearpage

\subsection{Partielle Integration}

\begin{theorem}[Partielle Integration]
	\[
		\int_a^b f(x)g'(x)dx
		= f(x)g(x)\Bigl|_a^b - \int_a^b f'(x)g(x) dx
	\]
\end{theorem}
\begin{proof}
	Folgt aus der \textbf{Produktregel}
	\begin{equation}
		\label{eq: product rule}
		\frac{d}{dx}f(x)g(x) = f(x) g'(x) + f'(x)g(x),
	\end{equation}
	da mit Umstellen gilt:
	\begin{equation*}
		\int_a^b f(x)g'(x)dx
		\overset{\eqref{eq: product rule}}= \int_a^b \frac{d}{dx}f(x)g(x) - f'(x)g(x) dx
		= f(x)g(x)\Bigl|_a^b - \int_a^b f'(x)g(x) dx.
		\qedhere
	\end{equation*}
\end{proof}


\begin{example}[\(\sin\)-\(\exp\) Produkt]\label{ex: sin-exp Product}
	Zeige
	\[
		I=\int_0^b \sin(t)\exp(-ct)dt = 
		\frac{1-(\cos(b) + c\sin(b))e^{-cb}}{1+c^2}.
	\]	
\end{example}
\begin{proof}
	\begin{align*}
		I 
		&= -\cos(t)\exp(-ct)\Bigl|_0^b - c\int\cos(t)\exp(-ct)dt\\
		&= \cos(0) - \cos(b)\exp(-cb) - c\left(
			\sin(t)\exp(-ct)\Bigl|_0^b + c\int\sin(t)\exp(-ct)dt
		\right)\\
		&= 1- \cos(b)e^{-cb} - c\left(
			\sin(b)e^{-cb} + c I
		\right)
	\end{align*}
	Impliziert nach umstellen nach \(I\)
	\[
		I = \frac{1-\cos(b)e^{-cb} -c\sin(b)e^{-cb}}{1+c^2}.
		\qedhere
	\]
\end{proof}

\begin{example}[\(\sin\)-Potenzen]
	Zeige
	\[
		I_n=\int_0^\pi\sin^n(t)dt
		= \frac{(n-1)!!}{n!!}\begin{cases}
			\pi & n\text{ gerade}\\
			2 & n\text{ ungerade}
		\end{cases}
	\]
	mit \(0!!=(-1)!!=1\) und \(n!! = n(n-2)!!\).
\end{example}
\begin{proof}
	Wir wenden Induktion an. Der Induktionsanfang ist \(I_0=\pi\) und \(I_1=2\)
	nach Beispiel~\ref{ex: sin integration}. Angenommen wir haben nun
	\(I_{n-1},I_n\), dann gilt
	\[\begin{aligned}
		I_{n+1}
		=\int_0^\pi\sin^{n+1}(t)dt
		&= \int_0^\pi\sin^n(t)\frac{d}{dt}\left(-\cos(t)\right)dt\\
		&= \underbrace{-\sin^n(t)\cos(t)\Bigl|_0^\pi}_{=0}
		- \int_0^\pi -n\sin^{n-1}(t)\underbrace{\cos^2(t)}_{1-\sin^2(t)}dt\\
		&= nI_{n-1} - nI_{n+1}.
	\end{aligned}
	\]
	Damit haben wir den Induktionsschritt:
	\[
		I_{n+1}
		= \frac{n}{n+1} I_{n-1}
		= \frac{n}{n+1}\frac{(n-2)!!}{(n-1)!!}
		\begin{cases}
			\pi & n-1 \text{ gerade}\\
			2 & n-1 \text{ ungerade}
		\end{cases}
		=\frac{n!!}{(n+1)!!}
		\begin{cases}
			\pi & n+1 \text{ gerade}\\
			2 & n+1 \text{ ungerade}.
		\end{cases}
		\qedhere
	\]
\end{proof}

\begin{example}[Gamma Funktion]
	Die Gamma Funktion ist definiert als
	\[
		\Gamma(z) = \int_0^\infty x^{z-1}\exp(-x)dx
	\]
	Es gilt \(\Gamma(n) = (n-1)!\) für alle \(n\in\nat\) und allgemeiner
	\(\Gamma(z+1) = z\Gamma(z)\). Die Gamma Funktion ist
	also eine Verallgemeinerung der Fakultät auf die reellen (komplexen) Zahlen.
\end{example}
\begin{proof}
	Die Aussage über die natürlichen Zahlen zweigen wir per Induktion mit
	Induktionsanfang \(n=1\)
	\[
		\Gamma(1) = \int_0^\infty \exp(-x)dx = \lim_{k\to\infty} -\exp(-x)\Bigl|_0^k
		= 1.
	\]
	Der Induktionsschluss folgt direkt aus der allgemeinen Aussage
	\(\Gamma(z+1)=z\Gamma(z)\), die wir nun zeigen:
	\begin{align*}
		\Gamma(z+1) &= \int_0^\infty x^z \exp(-x)dx\\
		\overset{\text{PI}}&=
		\underbrace{\lim_{k\to\infty} -x^z\exp(-x)\Bigl|_0^k}_{=0}
		- \int_0^k -zx^{z-1}\exp(-x)dx\\
		&= z\underbrace{\int_0^k -x^{n-1}\exp(-x)dx}_{=\Gamma(z)}.
		\qedhere
	\end{align*}
\end{proof}

Die Gammafunktion braucht man beispielsweise für die Gammaverteilung, die zum
Beispiel die Verteilung von Summen von quadrierten Normalverteilten
Zufallsvariablen oder Summen von Exponentiell verteilten Zufallsvariablen
beschreibt.


\begin{example}[Distributionen]
	Sei \(\varphi\in C^\infty_c\) (glatt mit kompaktem Support). Dann
	gilt für alle differenzierbaren \(f\)
	\begin{equation}\label{eq: ableitung vertauschen}
		\begin{split}
		\int f'(x) \varphi(x)dx 
		&=\underbrace{f(x)\varphi(x)\Bigl|_{-\infty}^\infty}_{
			=0\mathrlap{\quad\text{(kompakter Support)}}}
		- \int f(x)\varphi'(x) dx\\
		&= - \int f(x)\varphi'(x) dx.
		\end{split}
	\end{equation}
	D.h. wir können statt \(f\) auch immer \(\varphi\) ableiten. Man kann diesen
	Trick verwenden um nicht differenzierbare Funktionen ``abzuleiten''. Das
	wird verwendet, um Lösungen für PDEs mit unstetigen Anfangsbedingungen zu
	bestimmen. Diese Lösungen werden auch ``schwache Lösungen'' gennant, weil
	es sich dabei nicht um differenzierbare Funktionen, sondern
	``Distributionen'' handelt.
	
	Die Menge der Distributionen ist formal definiert als
	\(\cD:=(C^\infty_c)^*\), d.h. der Dualraum der glatten Funktionen mit
	kompaktem Support. Zur Erinnerung: der Dualraum
	enthält alle linearen Funktionen \(C^\infty_c\to\real\).
	Sei nun \(f\in C^\infty_c\), dann ist
	\[
		F:\begin{cases}
			C^\infty_c \to \real\\
			\varphi\mapsto \int \varphi(x)f(x) dx
		\end{cases}
	\]
	ein Element aus \(\cD\). Indem wir \(F\) mit \(f\) identifizieren
	wird \(C^\infty_c\) ein Unterraum von \(\cD\). Aber \(\cD\) ist noch viel
	größer. Alle stetigen Funktionen sind beispielsweise mit einer ähnlichen
 	Integral Identifizierung enthalten. Ebenso können wir ein Maß
	\(\mu\) mit der Distribution \(\varphi\mapsto \int \varphi(x)d\mu(x)\)
	identifizieren.
	
	Nun können wir Ableiten auf Distributionen definieren. Da für alle
	\(\varphi\in C^\infty_c\) gilt \(\varphi'\in C^\infty_c\), ist für alle \(F\in\cD\)
	\[
		F':= \frac{d}{dx} F := - F\circ \frac{d}{dx}
	\]
	wohldefiniert. Außerdem ist \(F'\) eine Distribution, da die Ableitung
 	linear ist, die Verknüpfung somit auch, und nach \(\real\) abbildet.
	Wegen \eqref{eq: ableitung vertauschen} ist die Ableitung von Distributionen
	auf dem Unterraum der differenzierbaren Funktionen identisch mit
	der normalen Ableitung. Somit handelt es sich wirklich um eine Erweiterung
	des Ableitungsbegriffs.

	Wir wollen nun zeigen, dass \(\ind_{[0,\infty)}'=\delta_0\) im Sinne von
	Distributionen. Mit anderen Worten: Wir können auch diskrete
	Verteilungsfunktionen ableiten um eine Art ``Dichte'' zu erhalten.

	Doch bevor
	wir das tun, wollen wir unsere Intuition für Gleichheit im Sinne von
	Distributionen schärfen. Denn im Sinne von Distributionen gilt
	\(\ind_{[0,\infty)}=\ind_{(0,\infty)}\). Warum ist das der Fall? Es gilt
	für jedes \(\varphi\in C^\infty_c\), dass
	\[
		\ind_{[0,\infty)}(\varphi)
		= \int \ind_{[0,\infty)}(x)\varphi(x) dx
		= \int \ind_{(0,\infty)}(x)\varphi(x)dx = \ind_{(0,\infty)}(\varphi)
	\]
	also handelt es sich um die selbe duale Abbildung. Allgemeiner tritt dieses
	Problem auf, wenn zwei Funktionen Lebesgue fast sicher gleich sind. Da wir
	bei stetigen Funktionen nicht einzelne Punkte versetzen können, haben wir
	dort dieses Problem dagegen nicht. Ein ähnliches Problem tritt bei der
	Definition der \(L^p\) Räume in Lemma~3.4.8 der Vorlesung auf.
	Das Problem wird dort mit Äquivalenzklassen gelöst. Hier könnte man so etwas
	auch tun.
	
	Zurück zu unserem ursprünglichen Ziel: es gilt für alle \(\varphi\in C^\infty_c\)
	\[
		\begin{aligned}
		\ind_{[0,\infty)}'(\varphi) 
		&= -\ind_{[0,\infty)}(\varphi')
		= -\int \ind_{[0,\infty)}(x)\varphi'(x)dx
		= -\varphi(x)\Bigl|_0^\infty\\
		&= \varphi(0)\\
		&= \int \varphi(x)d\delta_0(x)
		= \delta_0(\varphi).
		\end{aligned}
	\]
	Somit gilt \(\ind_{[0,\infty)}'=\delta_0\) im Sinne von Distributionen.
\end{example}

\clearpage

\subsection{Substitutionsregel}

\begin{theorem}[Substitutionsregel] Für ein differenzierbares \(g\), gilt
	\[
		\int_{g(a)}^{g(b)}f(y){\color{red} dy}
		= \int_a^b f(g(x)) {\color{cyan} g'(x)dx}
		= \int_a^b f(g(x)) {\color{cyan} \frac{dy}{dx}dx}
	\]
\end{theorem}
\begin{proof}
	Folgt aus der \textbf{Kettenregel}
	\begin{equation}\label{eq: chain rule}
		\frac{d}{dx} F(g(x)) = F'(g(x))g'(x).
	\end{equation}
	Denn sei \(F'=f\), dann gilt
	\begin{align*}
		\int_{g(a)}^{g(b)}f(y)dy
		&= F(g(b))-F(g(a))\\
		&= \int_a^b \frac{d}{dx} F(g(x))dx
		\overset{\eqref{eq: chain rule}}= \int_a^b f(g(x))g'(x)dx
		= \int_a^b f(g(x)) g'(x)dx.
		\qedhere
	\end{align*}
\end{proof}

\begin{remark}[Merkregel]
	So tun als sei \(\frac{dy}{dx}\) ein Bruch:
	\[
		y:=g(x) \implies \frac{dy}{dx} = g'(x) \implies \text{``}{\color{red}dy} = {\color{cyan}g'(x)dx}\text{''}
	\]
	Diese Intuition hilft bei der umgekehrten Verwendung. Sei \(g:=h^{-1}\), also
	\[
		y=h^{-1}(x) \implies \frac{dy}{dx} = \frac1{h'(h^{-1}(x))} = \frac1{h'(y)} = \frac1{\frac{dx}{dy}}
		\implies \text{``}{\color{red}h'(y)dy }= {\color{cyan}dx}\text{''}
	\]
	und definiere
	\(\tilde{f}(y) := f(y)h'(y)\), dann gilt
	\[
		\int_{h^{-1}(a)}^{h^{-1}(b)} \overbrace{f(y){\color{red}h'(y)}}^{=:\tilde{f}(y)}{\color{red}dy}
		= \int_{a}^{b} \tilde{f}(h^{-1}(x)) (h^{-1})'(x) dx
		= \int_{a}^{b} f(h^{-1}(x)) {\color{cyan} dx}.
	\]
\end{remark}
\begin{corollary}
	Wenn \(g\) zusätzlich invertierbar ist, dann gilt
	\begin{equation*}
		\int_{a}^{b} f(g(x)) {\color{cyan} dx}
		=\int_{g(a)}^{g(b)}f(y) {\color{red}(g^{-1})'(y) dy}
		=\int_{g(a)}^{g(b)}f(y) {\color{red} \frac{dx}{dy} dy}.
	\end{equation*}
\end{corollary}

\begin{example}[Parameter Entfernen]
	Mit der Substitutionsregel, lassen sich die Parameter der allgemeinen
	Exponentialverteilung entfernen:
	\[
		F_\lambda(t)
		= \int_0^t \underbrace{\lambda}_{=\color{cyan}\frac{dy}{dx}}\exp(-\lambda x){\color{cyan} dx}
		\overset{y=\lambda x}=
		\int_0^{\lambda t} \exp(-y){\color{red} dy}
		= F_1(\lambda t).
	\]
	Ähnlich kann man eine allgemeine Normalverteilung leicht auf \(\cN(0,1)\)
	zurückführen:
	\begin{align*}
		F_{\mu,\sigma^2}(t) &= \frac1{\sqrt{2\pi \sigma^2}}\int_{-\infty}^t
		\exp\left(-\frac{(x-\mu)^2}{2\sigma^2}\right){\color{cyan} dx}
		\overset{y=\frac{x-\mu}{\sigma}}=
		\frac1{\sqrt{2\pi \sigma^2}}\int_{-\infty}^{\frac{t-\mu}\sigma}
		\exp\left(-y/2\right)\underbrace{\color{red}\frac{dx}{dy}}_{=\sigma}{\color{red} dy}\\
		&= \frac1{\sqrt{2\pi}}\int_{-\infty}^{\frac{t-\mu}\sigma}
		\exp\left(-y/2\right)dy\\
		&= F_{0, 1}\left(\frac{t-\mu}\sigma\right).
	\end{align*}
\end{example}


\section{Lebesgue Integral}

An kaum keiner Stelle haben wir die Definition des Riemann-Integrals wirklich
benötigt. Wir brauchen nur Linearität und Monotonie für den Hauptsatz I und
bekommen dann den Hauptsatz II mit dem Mittelwertsatz quasi geschenkt dazu.
Alle folgenden Rechenregeln basieren dann wiederum nur auf den Hauptsätzen.
Die gleichen Aussagen gelten also auch für das Lebesgue Integral.
Folglich müssen beide Integrale für stetige Funktionen übereinstimmen
\[
	\int_{[a,b]} f d\lambda = \int_a^b f(x) dx.
\]
Mit der Linearität des Integrals bekommt man diese Aussagen dann letztendlich
auch für stückweise stetige Funktionen.

Die einzige Ausnahme, sind uneigentliche Integrale. Wann gilt also
\[
	\int_a^\infty f(x)dx = \int_{[a,\infty)} fd\lambda
\]
für (stückweise) stetige f? Das Riemann integral ist definiert als Limes von
endlichen Integralgrenzen (für die die Gleichheit gilt)
\[
	\int_a^\infty f(x)dx
	= \lim_{b\to\infty}\int_a^b f(x) dx
	= \lim_{b\to\infty}\int_{[a,b]} fd\lambda
	= \lim_{b\to\infty}\int \ind_{[a,b]} fd\lambda
\]
Nun fragt sich also, ob
\[
	\lim_{b\to\infty}\int \ind_{[a,b]} fd\lambda
	= \int \ind_{[a,\infty)} fd\lambda
\]
gilt. Ist \(f\ge0\) folgt das mit immer mit monotoner Konvergenz. Ist \(f\) bzw.
\(|f|\) integrierbar, folgt das immer mit dominierter Konvergenz.
\begin{example}[Riemann aber nicht Lebesgue]
	Berechne (und begründe die Existenz) des uneigentlichen Riemann Integrals
	\[
		\int_0^\infty \frac{\sin(x)}{x}dx = \frac{\pi}2.
	\]
	Zeige, dass \(\frac{\sin(x)}x\) nicht Lebesgue integrierbar ist.
\end{example}
\begin{proof}
	Nach Beispiel~\eqref{ex: exp integration} gilt \(\int_0^\infty e^{-xy}dy = \frac1x\),
	also
	\[
		\int_0^b \frac{\sin(x)}xdx
		= \int_0^b \int_0^\infty e^{-xy}\sin(x)dydx
		\overset{\text{Fubini}}= \int_0^\infty \underbrace{\int_0^b e^{-xy}\sin(x) dx}_{
			\overset{\text{Bsp.~\ref{ex: sin-exp Product}}}=
			\frac{1-(\cos(b) + y\sin(b))e^{-yb}}{1+y^2}
		} dy
	\]
	wobei wir hier Fubini verwenden können, da \(|e^{-xy}\sin(x)|\le e^{-xy}\)
	über \([0,b]\times[0,\infty)\) (auch Lebesgue) integrierbar ist. Nun gilt aber
	\[
		\left|\frac{1-(\cos(b) + y\sin(b))e^{-yb}}{1+y^2}\right|
		\le \frac1{1+y^2} + \frac{(1+y)e^{-yb}}{1+y^2} \le \frac{K}{1+y^2},
	\]
	da für \(b\ge 1\) gilt \(y e^{-yb} \le ye^{-y} \to 0\) für \(y\to\infty\)
	und damit uniform über all \(b\ge 1\) beschränkt ist. Damit haben wir eine
	integeribare\footnote{siehe \eqref{eq: integrate 1/(1+x^2)}}
	Majorante für dominierte Konvergenz gefunden und können den Limes über \(b\)
	in das Integral ziehen
	\[
		\lim_{b\to\infty}	
		\int_0^b \frac{\sin(x)}xdx
		= \int_0^b \lim_{b\to\infty}	\frac{1-(\cos(b) + y\sin(b))e^{-yb}}{1+y^2}dy
		= \int_0^b \frac1{1+y^2}dy.
	\]
	Wegen \((\tan^{-1})'(x)=\frac1{1+x^2}\), gilt aber nach dem Hauptsatz
	für Integral und Differentialrechnung
	\begin{equation}\label{eq: integrate 1/(1+x^2)}
		\int_0^\infty \frac1{1+x^2} dx
		= \tan^{-1}(x)\Bigl|_0^\infty
		= \lim_{x\to\infty} \tan^{-1}(x) - 0 = \frac{\pi}2.
	\end{equation}
	Bleibt noch zu zeigen, dass \(\frac{\sin(x)}x\) nicht Lebesgue integrierbar
	ist. Hierzu genügt es zu zeigen, dass
	\begin{align*}
		\int_{[0,\infty)} \left|\frac{\sin(x)}x\right| dx
		&\ge \int_{[0,\infty)} \frac{|\sin(x)|}{2\pi\lceil x/2\pi\rceil} dx
		= \sum_{k=1}^\infty \frac{1}{2\pi k} \int_{[2\pi (k-1), 2\pi k)} |\sin(x)|dx\\
		&= \underbrace{\frac1{2\pi}\int_0^{2\pi}|\sin(x)|dx}_{> 0} \underbrace{\sum_{k=1}^\infty \frac1k}_{=\infty}
		=\infty.
		\qedhere
	\end{align*}
\end{proof}


\subsection{Transformationssatz}

\begin{theorem}[Abstrakter Transformationssatz]
	Sei \(g:\Omega\to\Omega'\) messbar, \(f:\Omega'\to\real\) messbar und so dass
	folgendes wohldefiniert ist (\(f\ge 0\) oder integrierbar)
	\[
		\int_\Omega f\circ g {\color{cyan} d\mu} = \int_{\Omega'} f {\color{red} d\mu_g}
	\]	
	mit \(\mu_g(B):= \mu(g^{-1}(B))\).
\end{theorem}

Während der Beweis relativ einfach ist, ist das Ergebnis so auch noch nicht
sonderlich nützlich, weil \(\mu_g\) noch zu bestimmen ist.

\begin{lemma}[Lineares Bildmaß des Lebesgue Maß]\label{lem: lin pushforward lebesgue measure}
	Se \(\lambda\) das Lebesgue Maß auf \(\real^n\) und \(g\) eine invertierbare lineare Funktion, d.h.
	\[
		g(x) = Ax,
	\]
	dann gilt
	\[
		\lambda_g = \frac1{|\det(A)|}\lambda = |\det(A^{-1})|\lambda
	\]
\end{lemma}
\begin{proof}[Beweisskizze]
	\begin{enumerate}[wide, label=\textbf{Schritt \arabic*:}]
		\item 
		Zeige für jedes verschiebungsinvariantes Maß \(\mu\), dass es
		zwangsläufig von der Form \(c\lambda\) sein muss.

		Definiere hierfür
		\(c:=\mu([0,1]^n)\) und leite die Größe aller rationalen Quader durch
		Verschieben und zusammenstecken von kleineren Quadern aus dieser Konstante her.
		Da die Menge alle rationalen Quader ein schnittstabiles Erzeuger ist, der
		die Borel-\(\sigma\)-Algebra erzeugt, folgt mit dem Eindeutigkeitssatz,
		dass \(\mu=c\lambda\).

		\item Zeige, dass \(\lambda_g\) verschiebungsinvariant ist (folglich \(\lambda_g=c\lambda\)).

		Es gilt, für eine Borelmenge \(B\) und \(x\in\real^n\) 
		\[
			\lambda_g(B+x) = \lambda(g^{-1}(B+x))
			\overset{g^{-1} \text{linear}}=\lambda(g^{-1}(B)+g^{-1}(x))
			\overset{\lambda\text{ versch.inv.}}=\lambda(g^{-1}(B)) = \lambda_g(B).
		\]

		\item Bestimme die Konstante \(c\) durch einsetzen von \([0,1]^n\). Zuerst
		für diagonale \(A\).

		\(A^{-1}[0,1]^n\) streckt den Quader um \(1/a_{ii}\) in die Richtung \(i\).
		Falls \(a_{ii}<0\) flippt das Intervall außerdem auf die negative Seite,
		das hat allerdings keinen Einfluss auf die Größe.
		Das resultierende Volumen ist also
		\[
			\lambda_g([0,1]^n)=\lambda(A^{-1}[0,1]^n)
			= \bigl|\prod_{i=1}^n 1/a_{ii}\bigr| = \frac1{|\det(A)|}
		\]

		\item Verwende die Singulärwertzerlegung von \(A\), d.h. \(A=UDV^T\) mit
		\(U\), \(V\) Basiswechselmatrizen auf andere orthonormale Vektoren und \(D\)
		diagonal. \(U\) und \(V\) sind Isometrien (verändern das Volumen nicht).
		Also gilt
		\[
			\lambda_g([0,1]^n) = \frac1{|\det(D)|}.
		\]
		Zuletzt gilt für beliebige quadratische Matrizen \(C,B\)
		\[
			\det(CB)=\det(C)\det(B),
		\]
		also \(1=\det(\identity)=\det(UU^T)=\det(U)^2\) und somit
		\(\det(D)=\det(A)\).
		\qedhere
	\end{enumerate}
\end{proof}

\begin{theorem}[Transformationssatz in \(\real^n\)]
	Sei \(g:\Omega\to\Omega'\) ein Diffeomorphismus (stetig differenzierbar und invertierbar)
	mit \(\Omega,\Omega'\subseteq\real^n\), dann gilt
	\[
		\int_\Omega f\circ g(x) {\color{cyan}dx}
		= \int_{\Omega'} f(y) {\color{red}\bigl|\det\bigl((g^{-1})'(y)\bigr)\bigr| dy}
		= \int_{\Omega'} f(y) {\color{red}\bigl|\det\Bigl(\frac{dx}{dy}\Bigr)\bigr| dy}
	\]
	oder auch
	\[
		\int_\Omega f(y) {\color{red}dy}
		= \int_{\Omega'} f\circ g(x) {\color{cyan}\bigl|\det\bigl(g'(x)\bigr)\bigr| dx}
		= \int_{\Omega'} f\circ g(x) {\color{cyan}\bigl|\det\Bigl(\frac{dy}{dx}\Bigr)\bigr| dx}.
	\]
\end{theorem}
\begin{proof}[Beweisskizze]
	Zerteile \(\Omega\) durch ein \(\delta\) Gitter, sodass \(g\) dort bis auf
	ein \(\epsilon\) seiner ersten Taylorapproximation entpricht. Die erste
	Taylorapproximation ist aber eine affin-lineare Abbildung. Wir wenden also das
	Lemma~\ref{lem: lin pushforward lebesgue measure} für lineare Funktionen an,
	und stecke die Integrale vorsichtig wieder mit der Linearität von Integralen
	zusammen.

	Die zweite Gleichung ergibt sich aus \(\tilde{f}=f\circ g\) und \(\tilde{g}=g^{-1}\)
	und der ersten Formel.
\end{proof}

\begin{remark}
	Auch wenn dieser konkrete Transformationssatz sehr ähnlich zur Substitutionsregel
	aussieht, und in einer Dimension tatsächlich für invertierbare \(g\) mit ihr
	übereinstimmt\footnote{
		Invertierbarkeit von \(g\) impliziert, dass \(g'\) entweder strikt positiv
		oder strikt negativ ist. Und das Tauschen der Integralgrenzen bei
		monotonem Fallen führt zu einem Minus, was dem Betrag im
		Transformationssatz entspricht.
	}, ist er eigentlich nicht die Substitutionsregel in \(n\) Dimensionen.
	Denn da die Kettenregel auch in \(n\) Dimensionen gilt, ist die Substitutionsregel
	auch von sich aus schon in \(n\) Dimensionen gültig. Typischerweise möchte man
	aber lieber mit der Determinanten der Ableitung als der Ableitungsmatrix
	selbst arbeiten, weshalb die ``echte'' \(n\)-Dimensionale Substitutionsregel
	kaum Anwendung findet und dieser Transformationssatz häufig auch
	\(n\)-dimensionale Substitutionsregel genannt wird.
\end{remark}

\begin{example}[Polarkoordinaten]
	Ein besonderer Trick, der durch den Transformationssatz möglich wird, sind
	die Polarkoordinaten
	\[
		\phi:\begin{cases}
			[0,\infty)\times[0,2\pi) \to \real^2\\
			(r,\varphi) \mapsto r(\cos(\varphi),\sin(\varphi))
		\end{cases}
	\]
	Mit diesem Trick, lässt sich das Quadrat der Gaussdichte integrieren
	\[
		\begin{aligned}
			\left(\int \exp\bigl(-x^2/2\bigr)dx\right)^2
			&= \int \int \exp\left(-\frac{x^2+y^2}2\right) dxdy\\
			&= \int_0^{2\pi} \int_0^\infty \exp\Bigl(
				-\underbrace{\frac{(r\cos(\varphi))^2+(r\sin(\varphi))^2}2}_{=r^2/2}
			\Bigr)
			\underbrace{|\det(\phi'(r,\varphi))|}_{\overset{\eqref{eq: determinant}}=r} dr d\varphi
			\\
			&= \int_0^{2\pi} -\exp(-r^2/2)\Bigl|_0^\infty d\varphi\\
			&= 2\pi,
		\end{aligned}
	\]
	da gilt
	\begin{equation}\label{eq: determinant}
		\det(\phi'(r,\varphi))= \det\left(\begin{pmatrix}
			\cos(\varphi)	& -r\sin(\varphi)\\
			\sin(\varphi) & r\cos(\varphi)
		\end{pmatrix}\right) = r \cos(\varphi)^2 + r\sin(\varphi)^2 = r.
	\end{equation}
	Hier haben wir ein bisschen geschummelt: \(\phi\) ist kein Diffeomorphismus, da
	da \(\phi^{-1}\) nicht differenzierbar ist für \(\varphi\) nahe \(0\) bzw \(2\pi\).
	Um diesen Sprung zu entfernen, betrachte nur \((0,2\pi)\). Das entfernt nur
	eine Linie aus \(\real^2\), was Maß Null hat und hat somit keinen Effekt
	auf das Integral (richtiger Beweis ist Übung). 
\end{example}


\subsection{Vertauschung von Ableitung und Integral}

Wenn man den Hauptsatz der Analysis im Kopf hat, klingt das zunächst einmal
sinnlos. Aber wenn man eine andere Koordinate ableitet, als man integriert,
macht diese Frage durchaus Sinn.

\begin{theorem}[Ableitung und Integral Vertauschen]
	Sei \((t,\omega)\mapsto f(t,\omega)\) eine 
	\begin{itemize}[noitemsep,topsep=0pt]
		\item messbare Funktion,
		\item die über \(\omega\) \(\mu\)-integrierbar ist für alle feste \(t\),
		\item und für \(\mu\)-fast alle \(\omega\in\Omega\) gilt: \(t\to
		f(t,\omega)\) ist differenzierbar (bzw. absolutstetig) in \(t\).
	\end{itemize}
	Falls \(\frac{\partial}{\partial t}f\) ``lokal integrierbar in x'' ist, d.h.
	es \(a<b\in\real\) gibt, sodass \(x\in[a,b]\) und
	\[
		\int_a^b \int_\Omega
		\left|\frac{d}{dt}f(t,\omega)\right| d\mu(\omega)dt<\infty
	\]
	gilt, dann darf man Ableitung und Integral an der Stelle \(x\) vertauschen
	\[
		\frac{\partial}{\partial t}\int f(x, \omega)d\mu(\omega)
		= \int \frac{\partial}{\partial t}f(x,\omega)d\mu(\omega).
	\]
\end{theorem}
\begin{proof}
	Sei o.B.d.A. \(x<b\). Dann gilt	
	\begin{align*}
		\frac{\partial}{\partial t}\int f(x,\omega)d\mu(\omega)
		% &= \lim_{\epsilon\to 0}
		% \frac{\int f(x+\epsilon,\omega)d\mu(\omega) - \int f(x,\omega)d\mu(\omega)}{\epsilon}\\
		\overset{\text{linear}}&= \lim_{\epsilon\to 0}\int
		\frac{f(x+\epsilon,\omega) - f(x,\omega)}{\epsilon}d\mu(\omega)\\
		\overset{\text{Hauptsatz II}}&= \lim_{\epsilon\to 0}\int
		\frac1\epsilon \int_x^{x+\epsilon}\frac{\partial}{\partial t} f(t,\omega) dt d\mu(\omega)\\
		\overset{\text{Fubini}}&=
		\lim_{\epsilon\to 0}\frac1\epsilon \int_x^{x+\epsilon}
		\int \frac{\partial}{\partial t}f(t,\omega) d\mu(\omega)dt \\
		&=
		\frac{d}{dy}\Bigl|_{y=x} \int_x^{y}
		\int \frac{\partial}{\partial t}f(t,\omega) d\mu(\omega)dt\\
		\overset{\text{Hauptsatz I}}&=
		\int \frac{\partial}{\partial t}f(t,\omega) d\mu(\omega)dt.
		\qedhere
	\end{align*}
\end{proof}

Wie man sieht, brauchen wir eigentlich alle Voraussetzungen nur, damit die Objekte
über die wir sprechen wohldefiniert sind. Nur die ``lokalen Integrierbarkeit''
ist wirklich zusätzlich. Denn es genügt nicht, dass die Ableitung nur in \(x\) über \(\omega\) \(\mu\)-integrierbar
ist. Die resultierende Funktion in \(x\)
\[
	F: x\to \int \left|\frac{\partial}{\partial t}f(x,\omega)\right|d\mu(\omega)
\]
muss friedlich genug sein, dass man sie zumindest in einer Umgebung noch
integrieren kann. Die Folgenden Bedingungen sind also hinreichend:
\begin{enumerate}
	\item \(F\) stetig in \(x\)
	\item \(F\) beschränkt in \(x\)
	\item uniform \(\bigl|\frac{\partial}{\partial t}f(t,\omega)\bigr| \le
	g(\omega)\) für ein integrierbares \(g\)
	\item \(t\mapsto |\frac{\partial}{\partial t}f(t,\omega)|\)
	ist lokal um \(x\) monoton in \(t\).
\end{enumerate}

\begin{example}[Geometrische Verteilung]
	Sei \(\mu=\sum_{k=1}^\infty(1-p)^{k-1}p\delta_k\) die geometrische Verteilung,
	Dann gilt
	\[
		\int x d\mu(x) = \frac1p.
	\]
\end{example}
\begin{proof}
	Sei \(\nu(A):= |A\cap\nat|\) bzw. \(\nu=\sum_{k=1}^\infty \delta_k\) das Zählmaß auf den natürlichen Zahlen. Dies
	ermöglicht es uns, eine Summe als Integral aufzufassen:
	\begin{align*}
		\int x d\mu(x)
		&= \sum_{k=1}^\infty k(1-p)^{k-1}p
		= p \int \frac{d}{dp}-(1-p)^{k} d\nu(k)\\
		&= p \frac{d}{dp} -{\underbrace{\sum_{k=1}^\infty(1-p)^k}_{
				=\frac1{1-(1-p)} - 1
				\mathrlap{= \frac1p -1}
			}}
		= p\frac1{p^2} \\
		&= \frac1p.
		\qedhere
	\end{align*}
\end{proof}

\begin{example}[Stochastic Gradient Descent]
	Sei \(L:\theta\to \E[l(\theta, X)]\) eine Loss Funktion die wir minimieren wollen,
	für eine Zufallsvariable \(X\). Dann gilt
	\[
		\nabla L(\theta) = \nabla\E[l(\theta, X)]
		=\nabla \int l(\theta, x)d\Pr_X(x)
		=\int \nabla l(\theta, x)d\Pr_X(x)
		=\E[\nabla l(\theta, X)].
	\]
	\(\nabla l(\theta,X)\) ist also ein erwartungstreuer Schätzer für den Gradient von \(L\).
\end{example}

\clearpage

\section*{Integrationshilfe}

\begin{theorem*}[Partielle Integration]
	\[
		\int_a^b f(x)g'(x)dx
		= f(x)g(x)\Bigl|_a^b - \int_a^b f'(x)g(x) dx
	\]
\end{theorem*}
\begin{proof}
	Folgt aus der \textbf{Produktregel}.
\end{proof}

\begin{itemize}
	\item Funktioniert gut bei Produkten mit, oder Potenzen von \(\exp, \cos, \sin\)
	\item Wird verwendet zum Ableiten von Distributionen
\end{itemize}

\begin{theorem*}[Substitutionsregel]
	Für differenzierbare \(g\) gilt
	\begin{align*}
		\int_a^b f(g(x)) {\color{cyan} g'(x)dx}
		&=\int_{g(a)}^{g(b)}f(y){\color{red} dy}\\
	\intertext{ist g zusätzlich invertierbar, gilt}
		\int_{a}^{b} f(g(x)) {\color{cyan} dx}
		&=\int_{g(a)}^{g(b)}f(y) {\color{red}(g^{-1})'(y) dy}
		=\int_{g(a)}^{g(b)}f(y) {\color{red} \frac{dx}{dy} dy}.
	\end{align*}
\end{theorem*}
\begin{proof}
	Folgt aus der \textbf{Kettenregel}.
\end{proof}

\begin{itemize}
	\item Nützlich zum Entfernen von unnötigen Parametern
\end{itemize}

\begin{theorem*}[Transformationssatz in \(\real^n\)]
	Sei \(g:\Omega\to\Omega'\) ein Diffeomorphismus (stetig differenzierbar und invertierbar)
	mit \(\Omega,\Omega'\subseteq\real^n\), dann gilt
	\[
		\int_\Omega f\circ g(x) {\color{cyan}dx}
		= \int_{\Omega'} f(y) {\color{red}\bigl|\det\bigl((g^{-1})'(y)\bigr)\bigr| dy}
		= \int_{\Omega'} f(y) {\color{red}\bigl|\det\Bigl(\frac{dx}{dy}\Bigr)\bigr| dy}
	\]
	oder auch
	\[
		\int_\Omega f(y) {\color{red}dy}
		= \int_{\Omega'} f\circ g(x) {\color{cyan}\bigl|\det\bigl(g'(x)\bigr)\bigr| dx}
		= \int_{\Omega'} f\circ g(x) {\color{cyan}\bigl|\det\Bigl(\frac{dy}{dx}\Bigr)\bigr| dx}.
	\]
\end{theorem*}
\begin{proof}
	Bildmaß von linearen Funktionen bestimmen und abstrakten Trafo anwenden. Die
	erste Taylorapproximation ist auf immer kleineren Zerstückelungen von
	\(\Omega\) schlussendlich exakt.
\end{proof}

\begin{itemize}
	\item Besonderer Trick: Polarkoordinaten
\end{itemize}


\end{document} 
