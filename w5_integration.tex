% !TeX spellcheck = en_US
\documentclass[11pt]{article}
\usepackage{exerciseHead}

\input{Macros.tex} 

% Meta Information %%%%%%%%%%%%%%%%%%%%%%%%%%%%%%%%%%%

% included in all sheets (-> only one update per semester)
\course{Stochastik 1}
\tutorial{Gruppenarbeitstutorium}
\semester{HWS 2022}
\author{Felix Benning, Svenja Kaiser}
 % (file) for one update per semester
\tutorial{Merkblatt}
\author{Felix Benning}

\title{Integration}

%%%% Document %%%%%%%%%%%%%%%%%%%%%%%%%%%%%%%%%%%%%%%%
\begin{document}

\maketitle

\section{Hauptsatz der Analysis}

\begin{theorem}[Hauptsatz I]
	Sei
	\[
		F(x) = \int_a^x f(t) dt,
	\]	
	für ein {\color{red} stetiges} \(f\). Dann ist \(F\) stetig differenzierbar
	mit \(F'=f'\).
\end{theorem}
\begin{proof}
	\[
		F^{\delta}(x):=\frac{F(x+\delta) - F(x)}{\delta}
		= \frac1\delta \int_x^{x+\delta} f(t)dt
		\le \frac1\delta
		\int_x^{x+\delta}\sup_{s\in[x,x+\delta]} f(s)dt
		= \sup_{s\in[x,x+\delta]} f(s)
	\]
	Analog bekommt man die untere Schranke, mit der man mithilfe des
	Sandwichtheorems den Beweis beenden kann:
	\begin{align*}
		f(x)
		\overset{\text{stetig}}&= \liminf_{\delta\to 0}
		\inf_{s\in[x,x+\delta]} f(x)
		\le \liminf_{\delta\to 0} F^\delta(x)\\
		&\le F'(x)\\
		&\le \limsup_{\delta\to 0} F^\delta(x)
		\le \limsup_{\delta\to 0} \sup_{s\in[x,x+\delta]} f(x)
		\overset{\text{stetig}}= f(x).
		\qedhere
	\end{align*}
\end{proof}

\begin{theorem}[Hauptsatz II]
	Sei \(F\) differenzierbar mit \(F'=f\) für ein Riemann {\color{red}
	integrierbares} \(f\). Dann gilt
	\[
		F(b) - F(a) = \int_a^b f(t)dt.
	\]	
\end{theorem}
\begin{proof}
	Definiere eine Diskretisierung \(a=x_0\le \dots \le x_{n}=b\)
	\begin{equation*}
		F(b)-F(a) = \sum_{k=1}^n F(x_{k}) - F(x_{k-1})
		\overset{\text{\color{red}MWS}}= \sum_{k=1}^n f(\xi_k)\Delta x_k
		\le \sum_{k=1}^n\sup_{c\in[x_{k-1}, x_k]} f(c) \Delta x_k
	\end{equation*}	
	Da \(f\) Riemann integrierbar ist, gilt für \(\Delta x = \sup_k \Delta x_k\)
	\[
		\lim_{\Delta x \to 0}\sum_{k=1}^n\sup_{c\in[x_{k-1}, x_k]} f(c) \Delta x_k
		=\int_a^b f(x)dx.
	\]
	Analog bekommt man die untere Schranke.
\end{proof}

\section{Partielle Integration}

\begin{theorem}[Partielle Integration]
	\[
		\int_a^b f(x)g'(x)dx
		= f(x)g(x)\Bigl|_a^b - \int_a^b f'(x)g(x) dx
	\]
\end{theorem}
\begin{proof}
	Folgt aus der \textbf{Produktregel}
	\[
		\frac{d}{dx}f(x)g(x) = f(x) g'(x) + f'(x)g(x),
	\]
	da mit Umstellen gilt:
	\begin{equation*}
		\int_a^b f(x)g'(x)dx
		= \int_a^b \frac{d}{dx}f(x)g(x) - f'(x)g(x) dx
		= f(x)g(x)\Bigl|_a^b - \int_a^b f'(x)g(x) dx.
		\qedhere
	\end{equation*}
\end{proof}


\begin{example}[Gamma Funktion]
	Die Gamma Funktion ist definiert als
	\[
		\Gamma(z) = \int_0^\infty x^{z-1}\exp(-x)dx
	\]
	Es gilt \(\Gamma(n) = (n-1)!\) für alle \(n\in\nat\) und allgemeiner
	\(\Gamma(z+1) = z\Gamma(z)\). Die Gamma Funktion ist
	also eine Verallgemeinerung der Fakultät auf die reellen (komplexen) Zahlen.
\end{example}
\begin{proof}
	Die Aussage über die natürlichen Zahlen zweigen wir per Induktion mit
	Induktionsanfang \(n=1\)
	\[
		\Gamma(1) = \int_0^\infty \exp(-x)dx = \lim_{k\to\infty} -\exp(-x)\Bigl|_0^k
		= 1.
	\]
	Der Induktionsschluss folgt direkt aus der allgemeinen Aussage
	\(\Gamma(z+1)=z\Gamma(z)\), die wir nun zeigen:
	\begin{align*}
		\Gamma(z+1) &= \int_0^\infty x^z \exp(-x)dx\\
		\overset{\text{PI}}&=
		\underbrace{\lim_{k\to\infty} -x^z\exp(-x)\Bigl|_0^k}_{=0}
		- \int_0^k -zx^{z-1}\exp(-x)dx\\
		&= z\underbrace{\int_0^k -x^{n-1}\exp(-x)dx}_{=\Gamma(z)}.
		\qedhere
	\end{align*}
\end{proof}

Die Gammafunktion braucht man beispielsweise für die Gammaverteilung, die zum
Beispiel die Verteilung von Summen von quadrierten Normalverteilten
Zufallsvariablen oder Summen von Exponentiell verteilten Zufallsvariablen
beschreibt.

\section{Substitutionsregel}

\begin{theorem}[Substitutionsregel] Für ein differenzierbares \(g\), gilt
	\[
		\int_{g(a)}^{g(b)}f(y)dy= \int_a^b f(g(x)) g'(x)dx
	\]
\end{theorem}
\begin{proof}
	Folgt aus der \textbf{Kettenregel}
	\begin{equation}\label{eq: chain rule}
		\frac{d}{dx} F(g(x)) = F'(g(x))g'(x).
	\end{equation}
	Denn sei \(F'=f\), dann gilt
	\begin{align*}
		\int_{g(a)}^{g(b)}f(y)dy
		&= F(g(b))-F(g(a))\\
		&= \int_a^b \frac{d}{dx} F(g(x))dx
		\overset{\eqref{eq: chain rule}}= \int_a^b f(g(x))g'(x)dx
		= \int_a^b f(g(x)) g'(x)dx.
		\qedhere
	\end{align*}
\end{proof}

\begin{remark}[Physikermerkregel]
	So tun als sei \(\frac{dy}{dx}\) ein Bruch:
	\[
		y:=g(x) \implies \frac{dy}{dx} = g'(x) \implies \text{``}{\color{red}dy} = {\color{cyan}g'(x)dx}\text{''}
	\]
	Diese Intuition hilft bei der umgekehrten Verwendung. Sei \(g=h^{-1}\)
	\[
		y:=h^{-1}(x) \implies \frac{dy}{dx} = \frac1{h'(h^{-1}(x))} = \frac1{h'(y)}
		\implies \text{``}{\color{red}h'(y)dy }= {\color{cyan}dx}\text{''}
	\]
	und definiere \(a:=h^{-1}(\tilde{a})\), \(b:=h^{-1}(\tilde{b})\) und
	\(\tilde{f}(y) := f(y)h'(y)\), dann gilt
	\[
		\int_{a}^{b}f(y) {\color{red}h'(y) dy}
		= \int_{h^{-1}(\tilde{a})}^{h^{-1}(\tilde{b})} \tilde{f}(y) dy
		= \int_{\tilde{a}}^{\tilde{b}} \tilde{f}(h^{-1}(x)) (h^{-1})'(x) dx
		= \int_{h(a)}^{h(b)} f(h^{-1}(x)) {\color{cyan} dx}.
	\]
\end{remark}
\begin{corollary}
	Wenn \(g\) zusätzlich invertierbar ist, dann gilt
	\begin{align*}
		\int_a^b f(y){\color{red}dy}
		&= \int_{g^{-1}(a)}^{g^{-1}(b)} f(g(x)){\color{cyan}g'(x)dx}
		= \int_{g^{-1}(a)}^{g^{-1}(b)} f(g(x)){\color{cyan}\frac{dy}{dx}dx}\\
		\int_{a}^{b}f(y) {\color{red} \frac{dx}{dy} dy}
		=\int_{a}^{b}f(y) {\color{red}(g^{-1})'(y) dy}
		&= \int_{g^{-1}(a)}^{g^{-1}(b)} f(g(x)) {\color{cyan} dx}.
	\end{align*}
\end{corollary}

\section{Transformationssatz}

\section{Vertauschen von Differenzieren und Integrieren}

\clearpage

\section*{Merkblatt Integration}

\begin{theorem}[Partielle Integration]
	\[
		\int_a^b f(x)g'(x)dx
		= f(x)g(x)\Bigl|_a^b - \int_a^b f'(x)g(x) dx
	\]
\end{theorem}
\begin{proof}
	Folgt aus der \textbf{Produktregel}.
\end{proof}

\begin{itemize}
	\item Funktioniert gut bei Produkten mit, oder Potenzen von \(\exp, \cos, \sin\)
	\item Wird werwendet zum Ableiten von Distributionen (vgl. ``Intro PDE'')
\end{itemize}

\begin{theorem}[Substitutionsregel]
	\[
		\int_{g(a)}^{g(b)}f(y)dy= \int_a^b f(g(x)) g'(x)dx
	\]
\end{theorem}
\begin{proof}
	Folgt aus der \textbf{Kettenregel}.
\end{proof}



\end{document} 
