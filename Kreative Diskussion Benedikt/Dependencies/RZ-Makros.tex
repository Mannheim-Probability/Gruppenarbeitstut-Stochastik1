% RZ-Makros.TEX
% Dies sind Makros, die f�r alle mathematischen Texte gut sind.

\sloppy
\renewcommand{\thefootnote}{\fnsymbol{footnote}}
\renewcommand{\labelenumi}{(\alph{enumi})}
\renewcommand{\labelenumii}{(\roman{enumii})}
\renewcommand{\labelenumiii}{(\arabic{enumiii})}

%  Allgemeine Makros

\newcommand{\R}{\mathrm{I\!R}}
\newcommand{\N}{\mathrm{I\!N}}
\newcommand{\HH}{\mathrm{I\!H}}
\newcommand{\F}{\mathrm{I\!F}}
\newcommand{\E}{\mathrm{I\!E}}
\newcommand{\K}{\mathrm{I\!K}}
\newcommand{\PP}{\mathrm{I\!P}}

\newcommand{\V}{\mathbb{V}}
\newcommand{\cB}{\mathcal{B}}

\newcommand{\Z}{\mathchoice {\hbox{$\sf\textstyle Z\kern-0.4em
Z$}}{\hbox{$\sf\textstyle Z\kern-0.4em Z$}}{\hbox{$\sf\scriptstyle
Z\kern-0.3em Z$}}{\hbox{$\sf\scriptscriptstyle Z\kern-0.2em Z$}}}

\newcommand{\Q}{\mathchoice {\setbox0=\hbox{$\displaystyle\rm
Q$}\hbox{\raise0.15\ht0\hbox to0pt{\kern0.4\wd0\vrule
height0.8\ht0\hss}\box0}}{\setbox0=\hbox{$\textstyle\rm
Q$}\hbox{\raise0.15\ht0\hbox to0pt{\kern0.4\wd0\vrule
height0.8\ht0\hss}\box0}}{\setbox0=\hbox{$\scriptstyle\rm
Q$}\hbox{\raise0.15\ht0\hbox to0pt{\kern0.4\wd0\vrule
height0.7\ht0\hss}\box0}}{\setbox0=\hbox{$\scriptscriptstyle\rm
Q$}\hbox{\raise0.15\ht0\hbox to0pt{\kern0.4\wd0\vrule
height0.7\ht0\hss}\box0}}}

\newcommand{\C}{\mathchoice {\setbox0=\hbox{$\displaystyle\rm
C$}\hbox{\hbox to0pt{\kern0.4\wd0\vrule
height0.9\ht0\hss}\box0}}{\setbox0=\hbox{$\textstyle\rm C$}\hbox{\hbox
to0pt{\kern0.4\wd0\vrule
height0.9\ht0\hss}\box0}}{\setbox0=\hbox{$\scriptstyle\rm C$}\hbox{\hbox
to0pt{\kern0.4\wd0\vrule
height0.9\ht0\hss}\box0}}{\setbox0=\hbox{$\scriptscriptstyle\rm
C$}\hbox{\hbox to0pt{\kern0.4\wd0\vrule height0.9\ht0\hss}\box0}}}

\newcommand{\OO}{\mathchoice {\setbox0=\hbox{$\displaystyle\rm
O$}\hbox{\hbox to0pt{\kern0.4\wd0\vrule
height0.9\ht0\hss}\box0}}{\setbox0=\hbox{$\textstyle\rm O$}\hbox{\hbox
to0pt{\kern0.4\wd0\vrule
height0.9\ht0\hss}\box0}}{\setbox0=\hbox{$\scriptstyle\rm O$}\hbox{\hbox
to0pt{\kern0.4\wd0\vrule
height0.9\ht0\hss}\box0}}{\setbox0=\hbox{$\scriptscriptstyle\rm
O$}\hbox{\hbox to0pt{\kern0.4\wd0\vrule height0.9\ht0\hss}\box0}}}

\newcommand{\scrA}{\mathcal{A}}
\newcommand{\scrB}{\mathcal{B}}
\newcommand{\scrC}{\mathcal{C}}
\newcommand{\scrD}{\mathcal{D}}
\newcommand{\scrE}{\mathcal{E}}
\newcommand{\scrF}{\mathcal{F}}
\newcommand{\scrG}{\mathcal{G}}
\newcommand{\scrH}{\mathcal{H}} % alt \Hscr
\newcommand{\scrI}{\mathcal{I}}
\newcommand{\scrJ}{\mathcal{J}}
\newcommand{\scrK}{\mathcal{K}}
\newcommand{\scrL}{\mathcal{L}}
\newcommand{\scrM}{\mathcal{M}}
\newcommand{\scrN}{\mathcal{N}}
\newcommand{\scrO}{\mathcal{O}}
\newcommand{\scrP}{\mathcal{P}}
\newcommand{\scrQ}{\mathcal{Q}}
\newcommand{\scrR}{\mathcal{R}}
\newcommand{\scrS}{\mathcal{S}}
\newcommand{\scrT}{\mathcal{T}}
\newcommand{\scrU}{\mathcal{U}}
\newcommand{\scrV}{\mathcal{V}} % alt \Vscr
\newcommand{\scrW}{\mathcal{W}}
\newcommand{\scrX}{\mathcal{X}}
\newcommand{\scrY}{\mathcal{Y}}
\newcommand{\scrZ}{\mathcal{Z}}


%% Es werden Fu�not-Symbolgleichungsnummerierung und Normal-Gleichungsnummerierung
%% eingef�hrt. Die Z�hler sind seqcounter und neqcouter. In jedem neuen Abschnitt
%% beginnt die Z�hlung von vorne
\newcounter{neqcounter}[section]
\newcounter{seqcounter}[section]
\newcommand{\numberequation}[1]
   {\renewcommand{\theequation}{\arabic{equation}}
    \setcounter{equation}{\value{neqcounter}}
    \begin{equation}#1\end{equation}\stepcounter{neqcounter}}
\newcommand{\symbolequation}[1]
   {\renewcommand{\theequation}{\fnsymbol{equation}}
    \setcounter{equation}{\value{seqcounter}}
    \begin{equation}#1\end{equation}\stepcounter{seqcounter}}

\newcommand{\diff}{\mathrm{d}}
\newcommand{\Diff}{\mathrm{D}}
\newcommand{\id}{\mathrm{id}}
\newcommand{\GL}{\mathrm{GL}}
\newcommand{\SL}{\mathrm{SL}}
\newcommand{\SO}{\mathrm{SO}}
\newcommand{\End}{\mathrm{End}}
\newcommand{\Alt}{\mathrm{Alt}}
\newcommand{\Spur}{\mathrm{Spur}}
\newcommand{\rk}{\mathrm{rk}}
\newcommand{\rg}{\mathrm{rg}}
\newcommand{\Spann}{\mathrm{Spann}}
\newcommand{\Top}{\mathrm{Top}}
\newcommand{\del}{\partial}
\newcommand{\ddel}[2]{\frac{\partial#1}{\partial#2}}
\newcommand{\ddeldel}[3]{\frac{\partial^2#1}{\partial#2\:\partial#3}}
\newcommand{\ddelsquare}[2]{\frac{\partial^2#1}{\partial{#2}^2}}
\newcommand{\grad}{\mathrm{grad}}
\newcommand{\hess}{\mathrm{hess}}
\newcommand{\Hess}{\mathrm{Hess}}
\newcommand{\Rp}{\R_+}
\newcommand{\eps}{\varepsilon}
\newcommand{\vi}{\varphi}
\newcommand{\vkap}{\varkappa}
\newcommand{\thet}{\vartheta}
\newcommand{\UL}{\mathchoice
{\setbox0=\hbox{$\displaystyle U$}
 \setbox1=\hbox{$\displaystyle l$}
 \hbox{\box0\kern-0.80\wd1\lower0.029\ht1\box1}}
{\setbox0=\hbox{$\displaystyle U$}
 \setbox1=\hbox{$\displaystyle l$}
 \hbox{\box0\kern-0.80\wd1\lower0.029\ht1\box1}}
{\setbox0=\hbox{$\scriptstyle U$}
 \setbox1=\hbox{$\scriptstyle l$}
 \hbox{\box0\kern-0.80\wd1\lower0.029\ht1\box1}}
{\setbox0=\hbox{$\scriptscriptstyle U$}
 \setbox1=\hbox{$\scriptscriptstyle l$}
 \hbox{\box0\kern-0.80\wd1\lower0.029\ht1\box1}}
}

\newcommand{\Uo}{\UL^o}
\def\fam#1#2#3#4{({#1}_{#2})_{#2=#3,\dotsc,#4}}
\def\zz#1#2#3{#1=#2,\dotsc,#3}
\newcommand{\intint}[2]{\{#1,\dotsc,#2\}}
\newcommand{\Vektor}[2]{({#1}_1,\dotsc,{#1}_{#2})} 

\newcommand{\qmq}[1]{\quad\mbox{#1}\quad}
\newcommand{\Menge}[2]{\{\,#1\,|\,#2\,\}}
\newcommand{\Abstand}[1]{\mbox{}\par\vspace{-#1mm}}

\def\bild#1#2#3#4{\leavevmode\vbox to #1{\vfil \hbox to #2{\special{picture #4
   scaled #3}\hfil}}}
   %Parametr: #1=H�he,#2=Breite,#3=Skalierung 0.1<f<10,#4=Bildname

%%  Zu den Rahmen:
%%  grRahmen ist mit rahmen aus dem Alalysisscript identisch.
%%  grRahmen geht fast �ber die ganze Textbreite
%%
%%  klRahmen ist aus kleinerRahmen aus den RZ-Makros von 1996 entstanden;
%%  seine H�he ist etwas vergr��ert worden,
%%  seine Breite ist dem Text angepasst, dieser darf nur eine Zeile sein
%%
%%  varRahmen ist eine Neusch�pfung: Man stellt eine Breite ein. Dann wird der
%%  in dem Rahmen wie in einer Minipage behandelt.
%%
\newcommand{\grRahmen}[1]{\begin{center}\setlength{\fboxrule}{0.8pt}
           \setlength{\fboxsep}{8pt}
              \fbox{\begin{minipage}{140mm}\rule{0mm}{5mm}\hspace*{-2mm} #1
                    \end{minipage}}\end{center}}

\newcommand{\klRahmen}[1]{\begin{center}\large\setlength{\fboxrule}{0.8pt}
                     \setlength{\fboxsep}{8pt}
                     \fbox{\rule[-2mm]{0mm}{7mm} #1 }
                     \normalsize\end{center}}

\newcommand{\varRahmen}[2]{\begin{center}\setlength{\fboxrule}{0.8pt}
           \setlength{\fboxsep}{8pt}
              \fbox{\begin{minipage}{#1}\rule{0mm}{5mm}\hspace*{-2mm} #2
                    \end{minipage}}\end{center}}

% die 1996-Version
%\newcommand{\kleinerRahmen}[1]{\begin{center}\large\setlength{\fboxrule}{0.8pt}
%                     \setlength{\fboxsep}{8pt}\fbox{#1}\normalsize\end{center}}

%% Randnotizen vom Juli 1997
\setlength{\marginparsep}{5pt}
\setlength{\marginparwidth}{30pt}
\newcommand{\marginlabel}[1]           
                 {\mbox{}\marginpar{\raggedleft\hspace{0pt}#1}}
\newcommand{\Ausrufezeichen}{\marginlabel{\raisebox{-1.2ex}{\Huge$\boldsymbol{!}$}}}
\newcommand{\Randbalken}[2]{\marginlabel{{\rule[#1]{0.5mm}{#2}}}}
\newcommand{\Zeigefinger}{\marginlabel{\raisebox{-0.8ex}{\LARGE\ding{43}}}}
\newcommand{\Blume}{\marginlabel{\raisebox{-0.6ex}{\LARGE\ding{94}}}}

%  Differentialgeometrie - Makros
\newcommand{\bigoperp}{\mathop{\bigcirc\raisebox{0.25em}
 {\hskip-0.53em\hbox{\vrule height1.0ex width0.04em}
  \hskip-0.77em\hbox{\vrule height0.04em width 0.8em}
  \hskip- 0.2em}}}

\renewcommand{\bigoplus}{\mathop{\bigcirc
  \raisebox{-0.22em}{\hskip-0.53em\hbox{\vrule height2.08ex width0.04em}
  \raisebox{ 0.48em}{\hskip-0.75em\hbox{\vrule height0.04em width 0.8em}}
  \hskip- 0.2em}}}

\def\NP#1#2#3{\mathchoice
  {{\textstyle{\prod\limits_{i=#2}^{#3}}{#1}_i}} 
  {{\textstyle{\prod_{i=#2}^{#3}}{#1}_i}}
  {}
  {}
   }
\def\DS#1#2#3{\bigoplus_{i=#2}^{#3}\!#1_i} 
\def\OS#1#2#3{\bigoperp_{i=#2}^{#3}#1_i} 
\def\WP#1#2#3{#1_0 \times_{#2}\NP {#1}{1}{#3}}
\def\TP#1#2#3#4{{\rule{0mm}{2ex}}^{#2}\NP{#1}{#3}{#4}}

\newcommand {\cinf}{\ensuremath{\mathrm{C}^{\infty}}}
\newcommand {\X}{\mathfrak{X}}
\newcommand{\Tensor}[3]{\mathfrak{T}^{(#2,#3)}(#1)}
\newcommand{\alphad}{\dot{\alpha}}
\newcommand {\g}[2]{\langle #1,#2\rangle}
\def\gg{\g{\cdot}{\cdot}}
\newcommand {\euc}[2]{\langle\!\langle #1,#2 \rangle\!\rangle}
\newcommand {\peuc}[2]{\langle\!\langle #1,#2 \rangle\!\rangle\!_s}
\newcommand {\Kov}[2]{\nabla_{#1}#2}
\newcommand {\Kovh}[2]{\widehat{\nabla}_{#1}#2}
\newcommand {\Kovt}[2]{\widetilde{\nabla}_{#1}#2}
\newcommand {\Kovperp}[2]{\nabla^{\perp}_{#1}#2}
\newcommand {\Kovv}[3]{{\nabla^{\scriptscriptstyle{#1}}}_{\!#2}#3}
\newcommand {\operp}{\mathbin{\mbox{$\ominus\raisebox{2.9pt}
 {\hskip-0.42em\hbox{\vrule height0.7ex width0.02em}\hskip0.42em }$}}}
\newcommand {\TpM}{T_{p}M}
\newcommand {\Tpf}{T_{p}f}
\newcommand {\Nf}{\perp\!\!(f)}
\newcommand {\NM}{\perp\!\!M}
\newcommand {\Npf}{\perp_p\!\!(f)}
\newcommand {\NpM}{\perp_p\!\!M}
\newcommand {\Nepf}{\perp^{\!\!1}_p\!\!(f)}
\newcommand {\NepM}{\perp^{\!\!1}_p\!\!M}
\newcommand{\Shop}[3]{{\mathrm{A}^{^{\scriptscriptstyle{\!\!#1}}}_{#2}}#3} %shape op

% Horizontale/vertikale VR 
%\def\Vscr{\mathcal{V}}   ersetzt durch \scrV
%\def\Hscr{\mathcal{H}}   ersetzt durch \scrH
%\def\Vscrt{\widetilde{\mathcal{V}}}  ersetzt durch
\def\tscrV{\widetilde{\mathcal{V}}}
%\def\Hscrt{\widetilde{\mathcal{H}}}  ersetzt durch
\def\tscrH{\widetilde{\mathcal{H}}}
\newcommand{\hdisp}[3]{\overset{#2}{\underset{#1}{\parallel}}\!\!#3\,} 
    % horicontal displacement
\newcommand{\Hol}{\mathrm{Hol}}

% Reelle projektive R�ume
\newcommand {\RPn}{\ensuremath{\R\mathrm{P}^{n}}}
\newcommand {\RPm}{\ensuremath{\R\mathrm{P}^{m}}}

\newcommand {\cA}{\mathcal{A}}
% Komplexe Raumformen
\newcommand {\CPn}{\ensuremath{\C\mathrm{P}^{n}}}
\newcommand {\CPm}{\ensuremath{\C\mathrm{P}^{m}}}
\newcommand {\CHn}{\ensuremath{\C\mathrm{H}^{n}}}
\newcommand {\CHm}{\ensuremath{\C\mathrm{H}^{m}}}

\def\Mgf/{Mannigfaltigkeit}
\def\UMgf/{Untermannigfaltigkeit}
\def\Abb/{Abbildung}
\def\zshgd/{zusammenh�ngend}
\def\FR/{Faserraum}
\def\FB/{Faserb�ndel}
\def\VB/{Vektorb�ndel}
\def\UVB/{Untervektorb�ndel}

\hyphenation{Riemann-sche Riemann-ian}

\endinput
% Ende von RZ-Makros.TEX

