% !TeX spellcheck = en_US
\documentclass[11pt]{article}
\usepackage{exerciseHead}

\input{Macros.tex} 

% Meta Information %%%%%%%%%%%%%%%%%%%%%%%%%%%%%%%%%%%

% included in all sheets (-> only one update per semester)
\course{Stochastik 1}
\tutorial{Gruppenarbeitstutorium}
\semester{HWS 2022}
\author{Felix Benning, Svenja Kaiser}
 % (file) for one update per semester

\title{Important Probability Distributions}

%%%% Document %%%%%%%%%%%%%%%%%%%%%%%%%%%%%%%%%%%%%%%%
\begin{document}

\maketitle

\section{Useful Lemmas}

\begin{definition*}[Semigroup Homomorphism]
	A \emph{Semigroup homomorphism} \(f:(G,*)\to (H,\circ)\) respects the operation of the
	semigroups \((G,*), (H,\circ)\) in the sense of
	\[
		f(a * b) = f(a) \circ f(b)	\quad \forall a,b\in G.
	\]
	An \emph{endomorphism} is a semigroup homomorphism where \((G,*)=(H,\circ)\).
\end{definition*}

\begin{exercise}[Group Endomorphisms of \((\real, +)\)]
	For an arbitrary group endomorphism \(f\) on \((\real,+)\), i.e.
	\[
		f(a + b) = f(a) + f(b) \quad \forall a,b\in \real
	\]
	prove
	\begin{enumerate}
		\item \(f(n) = nf(1)\) for all \(n\in \nat\),
		\item \(f(q) = qf(1)\) for all \(q\in \rational\),
		\item \(f(x) = f(1)x\) for all \(x\in \real\) if \(f\) is continuous.
	\end{enumerate}
	So we have proven that the set of continuous endomorphisms on \((\real,+)\) is limeted to
	\[
		\{ x \mapsto ax : a \in \real\}
	\]
	\begin{enumerate}[resume]
		\item To understand why ``continuous'' is a necessary requirement, prove that
		\(\real\) is a vector space over \(\rational\). This implies that there exists a
		basis \(S\) (which includes \(1\) without loss of generality), such that any
		\(r\in \real\) can be uniquely represented by
		\[
			r = q_1 1 + q_2 s_2 + \dots q_n s_n	
		\]
		for some finite \(n\in\nat\), basis elements \(s_i \in S\) and coefficients \(q_i\in\rational\).
		Deduce that
		\[
			f(r) = q_1
		\]
		is a discontinuous endomorphism on \(\real\).
	\end{enumerate}
\end{exercise}

\begin{exercise}[Group Homomorphisms from \((\real,+)\) to \((\real,\cdot)\)]
	Prove that the set of continous group homomorphisms from \((\real,+)\) to
	\((\real,\cdot)\), i.e. \(f\) such that
	\[
		f(a+b) = f(a)f(b)\quad \forall a,b\in \real
	\]
	is given by
	\[
		\{ x\mapsto \exp(ax): a\in \real\}.
	\]
\end{exercise}

\begin{exercise}
	Characterize all continuous monoid (semigroup with identity) homeomorphisms
	\begin{enumerate}
		\item from \((\real_{\ge 0}, +)\) to \((\real_{\le 0}, +)\),
		\item from \((\real_{\ge 0}, +)\) to \(([0,1], \cdot)\).
	\end{enumerate}
\end{exercise}

\section{Distributions}

\begin{exercise}[Memoryless distributions]
	If the likelihood of an event happening later than \(t+s\) is given by the
	likelihood of it happening later than \(s\) multiplied by likelihood of it
	happening later than \(t\)
	\begin{equation}\label{eq: memoryless}
		\Pr(X>t+s) = \Pr(X>s)\Pr(X>t).
	\end{equation}
	we call the random arrival time \(X\) meomryless. Because
	the likelihood of taking another time \(s\) given that it already took time \(t\)
	remains the same (\(X\) has no memory), i.e.
	\begin{equation}
		\label{eq: memoryless with condition}
		\Pr(X > t+s \mid X > t) = \Pr(X > s)
	\end{equation}
	\begin{enumerate}
		\item Prove that \eqref{eq: memoryless with condition} is actually equivalent to \eqref{eq: memoryless}.
		\item Prove that there is only one continuous memoryless distribution on
		\(\real_+\) and determine its cumulative distribution function.
		\item Prove that there is only one discrete meomryless distribution on the natural numbers
		and represent it as a series of dirac measures.
		\item Let \(X\) be a continuous memoryless random variable. Prove that
		\(\lfloor X \rfloor\) is a memoryless discrete random vairable. Represent the
		parameters of the discrete distribution as a function of the continuous distribution.
	\end{enumerate}

	Example: In a discrete time setting this might be throwing a dice and waiting for the
	first time \(6\) is rolled. If it was not rolled after \(3\) rolls, then the likelihood
	of it happening within the next two rolls is the same as if we did not roll
	\(3\) dice before.
\end{exercise}

\begin{exercise}[Rotation invariant distribution]
	Assume that the likelihood of hitting a point \(x\) only depends on its
	distance to the origin \(0\) (i.e. it is rotation invariant). This implies
	we can write our density as
	\[
		f(x_1,\dots, x_n) = \tilde{g}\Bigl(\sqrt{x_1^2 + \dots+ x_n^2}\Bigr)
		= g\bigl(x_1^2 +\dots + x_n^2\bigr)
	\]
	Additionally assume that the magnitude of the coordinates \(X_i\) of the random
	point \(X\) are independent. This implies we can write our density as
	\[
		f_1(x_1)\cdots f_2(x_n) = f(x_1,\dots, x_n)
	\]
	\begin{enumerate}
		\item Prove
		\[
			f_i(x_i) = c_i g(x_i^2)
		\]
		for some \(c_i\in\real\)

		\item Deduce
		\[
			\underbrace{\Bigl(\prod_{i=1}^n c_i\Bigr)}_{=c} g(x_1^2)\cdots g(x_n^2)
			= g(x_1^2 + \dots + x_n^2)
		\]
		Prove that either \(g(0)=0\) or \(c=g(0)=1\).
		
		\item Prove that \(g(0)=0\) implies that \(g\equiv 0\) and thus it can not
		be a density.

		\item Set \(x_i = 0\) for all \(i>2\) and apply our characterization of
		homeomorphisms from \((\real_{\ge 0}, +)\) to \(([0,1], \cdot)\) to prove that
		\(f\) (if continuous) is the density of a centred normal distribution.
	\end{enumerate}
\end{exercise}

\end{document}