% !TeX spellcheck = en_US
\documentclass[11pt]{article}
\usepackage{exerciseHead}

\input{Macros.tex} 

% Meta Information %%%%%%%%%%%%%%%%%%%%%%%%%%%%%%%%%%%

% included in all sheets (-> only one update per semester)
\course{Stochastik 1}
\tutorial{Gruppenarbeitstutorium}
\semester{HWS 2022}
\author{Felix Benning, Svenja Kaiser}
 % (file) for one update per semester

\title{(In)dependence}

%%%% Document %%%%%%%%%%%%%%%%%%%%%%%%%%%%%%%%%%%%%%%%
\begin{document}

\maketitle

\begin{exercise}[Idependent vs Uncorrelated]
	Let \(X\) and \(Y\) be random variables with existing second moments.  Which of these statements are
	equivalent? There can be multiple equivalence groups. How do they relate to
	each other? Does one imply the other?
	\begin{enumerate}
		\item \(X\) and \(Y\) are independent.
		\item \(X\) and \(Y\) are uncorrelated.
		\item for any measurable functions \(f,g\), \(f(X)\) and \(g(Y)\) are uncorrelated.
		\item for any linear functions \(f,g\) \(f(X)\) and \(g(Y)\) are uncorrelated.
		\item for any measurable functions \(f,g\), \(f(X)\) and \(g(Y)\) are independent.
		\item for any linear functions \(f,g\) \(f(X)\) and \(g(Y)\) are independent.
	\end{enumerate}
\end{exercise}
\begin{hint}
	You may want to assume \(X\) and \(Y\) are centered without loss of generality.
\end{hint}

\begin{exercise}[Wahrscheinlichekeitsbaum]
	Verwende 4.4.3 um einen Wahrscheinlichkeitsbaum zu formalisieren.
\end{exercise}


\end{document}