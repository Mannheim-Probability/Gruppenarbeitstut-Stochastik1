
% !TeX spellcheck = en_US
\documentclass[11pt]{article}
\usepackage{exerciseHead}

\input{Macros.tex} 

% Meta Information %%%%%%%%%%%%%%%%%%%%%%%%%%%%%%%%%%%

% included in all sheets (-> only one update per semester)
\course{Stochastik 1}
\tutorial{Gruppenarbeitstutorium}
\semester{HWS 2022}
\author{Felix Benning, Svenja Kaiser}
 % (file) for one update per semester

\title{Inclusion-Exclusion Prinzip}

%%%% Document %%%%%%%%%%%%%%%%%%%%%%%%%%%%%%%%%%%%%%%%
\begin{document}

\maketitle

Sei \((\Omega, \cA, \Pr)\) ein Wahrscheinlichkeitsraum.

\begin{exercise}[Zwei Mengen]
	Beweise für alle \(A,B\in\cA\)
	\[
		\Pr(A \cup B) = \Pr(A) + \Pr(B) - \Pr(A\cap B)
	\]
\end{exercise}

\begin{exercise}[n-Mengen]\label{ex: n-Mengen}
	Beweise für alle \(n\in\nat\), \(A_1,\dots, A_n\in\cA\)
	\[
		\Pr\left(
			\bigcup_{k=1}^n A_k
		\right)
		= \sum_{k=1}^n (-1)^{k+1} \sum_{I\subseteq \{1,\dots,n\}: |I|=k} \Pr(A_I)
	\]
	wobei wir definieren
	\[
		A_I := \bigcap_{i\in I} A_i
	\]
\end{exercise}

\begin{exercise}[Spezialfall]
	Im gleichen Setting wie Aufgabe~\ref{ex: n-Mengen} nehmen wir zusätzlich an,
	dass es \(p_1, \dots, p_n\) gibt mit
	\[
		\Pr(A_I) = p_{|I|},
	\]
	d.h. die Wahrscheinlichkeit von \(\Pr(A_I)\) nur von der Anzahl der Elemente
	in \(I\) abhängt.
	Zeige dass dann gilt:
	\[
		\Pr\left(\bigcup_{k=1}^n A_k\right) = \sum_{k=1}^n (-1)^{k+1} \binom{n}{k}p_k
	\]
\end{exercise}

\end{document} 