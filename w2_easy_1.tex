
% !TeX spellcheck = en_US
\documentclass[11pt]{article}
\usepackage{exerciseHead}

\input{Macros.tex} 

% Meta Information %%%%%%%%%%%%%%%%%%%%%%%%%%%%%%%%%%%

% included in all sheets (-> only one update per semester)
\course{Stochastik 1}
\tutorial{Gruppenarbeitstutorium}
\semester{HWS 2022}
\author{Felix Benning, Svenja Kaiser}
 % (file) for one update per semester


%%%% Document %%%%%%%%%%%%%%%%%%%%%%%%%%%%%%%%%%%%%%%%
\begin{document}


\begin{exercise*}[Nullmengen]
	Zeige, dass für ein beliebiges äußeres Maß \(\mu^*\) gilt, alle Nullmengen,
	\[
		A\subseteq \Omega : \mu^*(A) = 0,
	\]
	sind \(\mu^*\)-messbar.
\end{exercise*}

\begin{exercise*}[Mengenlehre]
	Zeige	
	\begin{enumerate}
		\item 
		\[
			(A\cap B)^\complement = A^\complement \cup B^\complement
		\]	
	
		\item
		\[
			A \cap (B\cup C) = (A\cap B) \cup (A \cap C)
		\]

		\item
		\[
			A \cup (B\cap C) = (A\cup B) \cap (A \cup C)
		\]

		\item
		\[
			B\setminus (B\setminus A) = A \cap B
		\]
	\end{enumerate}
\end{exercise*}

\begin{exercise*}[Potenzmenge]
	\begin{enumerate}
		\item 
		Was ist die Potenzmenge \(\cP(M)\) von \(M:=\{o,m,a\}\)?	

		\item
		Eine Menge besteht aus \(n\) Elementen. Wie viele Elemente enthält die
		Potenzmenge?
	\end{enumerate}

\end{exercise*}

\end{document} 
