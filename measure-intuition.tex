% !TeX spellcheck = en_US
\documentclass[11pt]{article}
\usepackage{exerciseHead}

\input{Macros.tex} 

% Meta Information %%%%%%%%%%%%%%%%%%%%%%%%%%%%%%%%%%%

% included in all sheets (-> only one update per semester)
\course{Stochastik 1}
\tutorial{Gruppenarbeitstutorium}
\semester{HWS 2022}
\author{Felix Benning, Svenja Kaiser}
 % (file) for one update per semester

\title{Intuition von Maßräumen}

%%%% Document %%%%%%%%%%%%%%%%%%%%%%%%%%%%%%%%%%%%%%%%
\begin{document}

\maketitle


\begin{exercise}[Das Lebesgue Maß]
	Diskutiert welche Eigenschaften ein Volumenmaß \(\lambda\) für den
	\(\real^n\) haben sollte.  Für \(n=2\) sollte das Volumenmaß ein
	``Flächenmaß'' sein, für
	\(n=1\) ein ``Längenmaß''. Versucht die Liste der Eigenschaften auf das
	nötigste zu reduzieren. Es sollten keine Eigenschaften übrig bleiben die
	sich aus den anderen ergeben.

	Vergleicht eure Definition eines Volumenmaßes mit der Definition eines
	allgemeinen Maßes in der Vorlesung. Versucht Gemeinsamkeiten und Unterschiede
	zu finden und zu begründen.
\end{exercise}

\begin{hint}
	Wie verändert sich das Volumen einer Menge, wenn sie verschoben wird?
	Wie bestimmt man den Flächeninhalt eines Quaders, eines Kreises?
\end{hint}

\begin{hint}
	Um den Flächeninhalt eines Kreises zu bestimmen, muss man den Grenzwert
	von immer feineren Approximationen ziehen können. Welche Eigenschaft braucht
	ein Maß damit das erlaubt ist?
\end{hint}

\begin{hint}
	Gegeben \(\lambda([0,a]^n) = c\), lässt sich mit Verschiebungsinvarianz auch
	das Volumen von \(\lambda([0,2a]^n)\) bestimmen? Wie sieht es mit allgemeinen
	Quadern aus?
\end{hint}

\begin{hint}
	Versucht Satz 1.2.12 zu verwenden um zu zeigen, dass die folgenden
	Eigenschaften das Lebesgue-Maß eindeutig bestimmen:
	\begin{itemize}
		\item \(\lambda(\emptyset) = 0, \lambda(A)\ge 0\) für alle \(A\)
		\item \(\lambda(\biguplus_{k=1}^\infty A_k) = \sum_{k=1}^\infty \lambda(A_k)\)
		``Approximierbarkeit''
	\end{itemize}
	So weit haben wir nur ein allgemeines Maß wie in der Vorlesung definiert.
	Jetzt kommen noch die zusätzlichen Eigenschaften:
	\begin{itemize}
		\item \(\lambda(A+x) = \lambda(A)\) für alle \(x\in\real^n\)
		``Verschiebungsinvarianz''
		\item \(\lambda([0,1]^n) = 1\) ``Normierung''
	\end{itemize}
	Wir lassen hier "Messbarkeit" von Mengen (\(\sigma\)-Algebren) unter den
	Tisch fallen. Nehmt einfach an, dass die Borel \(\sigma\)-Algebra ``alle''
	Mengen sind. Zunächst einmal möchte man ja vielleicht alle Mengen messen
	können. In Aufgabe~\ref{ex: why measurability} erkunden wir, warum das nicht
	geht.
\end{hint}

\begin{solution}
	Durch Verschiebungsinvarianz muss für \(\lambda([0, 1/m]^n)=c\) gelten
	\(\lambda([0,1]^n) = m^n c\). Folglich gilt \(c=\frac{1}{m^n}\). Mit
	ähnlichen Tricks, kann man \(\lambda([q_1, p_1]\times ...\times [q_n, p_n])\)
	für alle \(q_i,p_i\in\rational\) bestimmen. Aber \(\{\prod_{i=1}^n[q_i, p_i]
	: q_i,p_i\in\rational\}\) ist ein schnittstabilier Erzeuger der Borel
	\(\sigma\)-Algebra. Damit ist \(\lambda\) nach Satz 1.2.12 auf den
	Borelmengen eindeutig.
\end{solution}


\newpage

\begin{exercise}[Warum Messbarkeit]\label{ex: why measurability}	
	In dieser Aufgabe wollen wir diskutieren, warum wir \(\sigma\)-Algebren
	(Mengen von ``Messbaren'' Mengen) brauchen. Das Ziel ist es also nicht-messbare
	Mengen konstruieren, d.h. Mengen denen man nicht sinnvoll eine Größe zuordnen
	kann.
\end{exercise}
\begin{hint}
	
\end{hint}

\end{document} 