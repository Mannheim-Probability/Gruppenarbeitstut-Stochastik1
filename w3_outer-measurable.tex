% !TeX spellcheck = en_US
\documentclass[11pt]{article}
\usepackage{exerciseHead}

\input{Macros.tex} 

% Meta Information %%%%%%%%%%%%%%%%%%%%%%%%%%%%%%%%%%%

% included in all sheets (-> only one update per semester)
\course{Stochastik 1}
\tutorial{Gruppenarbeitstutorium}
\semester{HWS 2022}
\author{Felix Benning, Svenja Kaiser}
 % (file) for one update per semester

\title{Äußeres Maß}

%%%% Document %%%%%%%%%%%%%%%%%%%%%%%%%%%%%%%%%%%%%%%%
\begin{document}

\maketitle

\begin{exercise}[Nullmengen]
	Zeige, dass für ein beliebiges äußeres Maß \(\mu^*\) gilt, alle Nullmengen,
	\[
		A\subseteq \Omega : \mu^*(A) = 0,
	\]
	sind \(\mu^*\)-messbar.
\end{exercise}
\begin{exercise}
	Zeige, dass für jede Menge \(A\subseteq \real\) mit positivem äußerem
	Lebesgue Maß \(\lambda^*(A)\ge 0\) eine nicht \(\lambda^*\)-messbare
	Teilmenge existiert.
\end{exercise}
\begin{hint}
	Es gibt ein Intervall \(I\) mit \(\lambda^*(A\cap I)\ge 0\). Sei ohne
	Einschränkung \(A=A\cap I\) und schneide \(A\) mit Vitali Mengen. Zuletzt
	Subadditivität verwenden.
\end{hint}

\begin{remark}
	Da wir das Lebesgue Maß \(\lambda\) über ein äußeres Maß konstruieren, gilt
	also dass alle Nullmengen ``Lebesgue messbar'' sind, d.h. in
	\(\cA_{\lambda^*}\) enthalten sind. In ``Intuition von Maßräumen'' haben wir
	Vitali-Mengen konstruiert die nicht Lebesgue messbar sein können. Wir
	wollen nun Nullmengen finden, die nicht Borel messbar sind um zu zeigen
	\[
		\cB(\real^n)
		\overset{\text{Ziel}}{\subsetneq} \cA_{\lambda^*}
		\overset{\text{Vitali}}{\subsetneq} \cP(\real^n).
	\]
\end{remark}

\begin{exercise}[Cantor Menge]
	Zeige dass die ``Cantor Menge'' \(\cC\) (nachschlagen) Lebesgue Maß Null hat.
\end{exercise}

\begin{remark}
	Leider ist \(\cC\) aber Borel messbar. Um eine nicht Borel-messbare Nullmenge
	zu finden, wollen wir deren Teilmengen betrachten. Es gibt zwei Möglichkeiten
	die Existenz einer nicht Borel-messbaren Teilmenge zu beweisen.

	Die erste ist ein Kardinalitätsargument: Die Cantor Menge hat die gleiche
	Kardinalität wie das Kontinuum (ist überabzählbar). Also ist die Kardinalität
	ihrer Potenzmenge strikt größer als das Kontinuum (``größeres Unendlich'').
	Die Menge der Borel messbaren Mengen ist dagegen genaus groß wie das
	Kontinuum, also gibt es mehr Teilmengen der Cantor Menge als Borel Mengen.
	Daraus folgt, dass es eine Teilmenge geben muss, die keine Borel-Menge ist.

	Um ein bisschen näher an den Werkzeugen dieser Vorlesung zu bleiben,
	versuchen wir uns an einem etwas konstruktiveren Argument.
\end{remark}


\begin{exercise}[Cantor Funktion]
	Zeige dass die ``Cantor Funktion'' \(\phi:[0,1]\to [0,1]\) (nachschlagen) stetig und
	(schwach) monoton wachsend ist. Folgere, dass
	\[
		\psi : \begin{cases}
			[0,1] \to [0,2]\\
			x\mapsto x + \phi(x)
		\end{cases}	
	\]
	stetig, strikt monoton wachsend und stetig invertierbar ist.
\end{exercise}

\begin{exercise}[Finde Nicht Borel Messbare Menge]
	Beweise dass \(\psi(\cC)\) Lebesgue Maß \(1\) hat und folgere, dass es somit
	ein Lebesgue messbares \(B\subseteq \cC\) gibt, mit \(\psi(B)\) nicht
	Lebesgue (also auch nicht Borel messbar).
\end{exercise}

\begin{exercise}[Zielgerade]
	Zeige, dass stetige Abbildungen Borel Mengen auf Borel Mengen abbilden und
	folgere dass es Lebesgue messbare Mengen gibt, die nicht Borel messbar sind.	
\end{exercise}

\end{document} 
